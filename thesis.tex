\documentclass[12pt,oneside,a4paper,parskip]{scrbook}
\usepackage[utf8]{inputenc}
\usepackage{csquotes}
\usepackage[ngerman]{babel}
\usepackage{floatflt}
\usepackage{subfigure}
\usepackage[pdftex]{graphicx}
\usepackage[hidelinks]{hyperref}
\usepackage{color}
\usepackage{amssymb}
\usepackage{textcomp}
\usepackage{nicefrac}
\usepackage{scrhack}
\usepackage{pdfpages}
\usepackage{float}
\usepackage{pdflscape}
\usepackage{subfigure}
\usepackage{pdfpages}
\usepackage[verbose]{placeins}
\usepackage[headsepline,plainfootsepline]{scrlayer-scrpage}
\usepackage{listings}
\usepackage{xcolor}
\usepackage{color}
\usepackage{caption}
\usepackage{subfigure}
\usepackage{epstopdf}
\usepackage{longtable}
\usepackage{setspace}
\usepackage{booktabs}
\usepackage[style=numeric,backend=bibtex]{biblatex}
\bibliography{literatur}


%%%%%%%%%%%%%%%%%%%
%% definitions
%%%%%%%%%%%%%%%%%%%
\def\BaAuthor{Henning Janning}
\def\BaTitle{Evaluation von Web Application Security Testing Tools}
\def\BaSupervisorOne{Prof. Andreas Mayer}
\def\BaSupervisorTwo{Susanne Steuer (M.Sc.) }
\def\BaDeadline{05.04.2019}
\def\MatNr{192972}

\hypersetup{
pdfauthor={\BaAuthor},
pdftitle={\BaTitle},
pdfsubject={Subject},
pdfkeywords={Keywords}
}

%%%%%%%%%%%%%%%%%%%
%% configs to include
%%%%%%%%%%%%%%%%%%%
\colorlet{punct}{red!60!black}
\definecolor{background}{HTML}{EEEEEE}
\definecolor{delim}{RGB}{20,105,176}
\colorlet{numb}{magenta!60!black}

\definecolor{gray}{rgb}{0.4,0.4,0.4}
\definecolor{darkblue}{rgb}{0.0,0.0,0.6}
\definecolor{cyan}{rgb}{0.0,0.6,0.6}

\definecolor{pblue}{rgb}{0.13,0.13,1}
\definecolor{pgreen}{rgb}{0,0.5,0}
\definecolor{pred}{rgb}{0.9,0,0}
\definecolor{pgrey}{rgb}{0.46,0.45,0.48}

\lstset{
  basicstyle=\ttfamily,
  columns=fullflexible,
  showstringspaces=false,
  commentstyle=\color{gray}\upshape
  linewidth=\textwidth
}

\lstdefinelanguage{json}{
    basicstyle=\normalfont\ttfamily,
    numbers=left,
    numberstyle=\scriptsize,
    stepnumber=1,
    numbersep=8pt,
    showstringspaces=false,
    breaklines=true,
    backgroundcolor=\color{background},
    literate=
     *{0}{{{\color{numb}0}}}{1}
      {1}{{{\color{numb}1}}}{1}
      {2}{{{\color{numb}2}}}{1}
      {3}{{{\color{numb}3}}}{1}
      {4}{{{\color{numb}4}}}{1}
      {5}{{{\color{numb}5}}}{1}
      {6}{{{\color{numb}6}}}{1}
      {7}{{{\color{numb}7}}}{1}
      {8}{{{\color{numb}8}}}{1}
      {9}{{{\color{numb}9}}}{1}
      {:}{{{\color{punct}{:}}}}{1}
      {,}{{{\color{punct}{,}}}}{1}
      {\{}{{{\color{delim}{\{}}}}{1}
      {\}}{{{\color{delim}{\}}}}}{1}
      {[}{{{\color{delim}{[}}}}{1}
      {]}{{{\color{delim}{]}}}}{1},
}

\lstset{language=xml,
  morestring=[b]",
  morestring=[s]{>}{<},
  morecomment=[s]{<?}{?>},
  stringstyle=\color{black},
  numbers=left,
  numberstyle=\scriptsize,
  stepnumber=1,
  numbersep=8pt,
  identifierstyle=\color{darkblue},
  keywordstyle=\color{cyan},
  backgroundcolor=\color{background},
  morekeywords={xmlns,version,type}% list your attributes here
}

\lstset{language=Java,
  showspaces=false,
  showtabs=false,
  tabsize=4,
  breaklines=true,
  keepspaces=true,
  numbers=left,
  numberstyle=\scriptsize,
  stepnumber=1,
  numbersep=8pt,
  showstringspaces=false,
  breakatwhitespace=true,
  commentstyle=\color{pgreen},
  keywordstyle=\color{pblue},
  stringstyle=\color{pred},
  basicstyle=\ttfamily,
  backgroundcolor=\color{background},
%  moredelim=[il][\textcolor{pgrey}]{$$},
%  moredelim=[is][\textcolor{pgrey}]{\%\%}{\%\%}
}



\tracingmacros=1
\begin{document}
\tracingmacros=0

%%%%%%%%%%%%%%%%%%%
%% Titelseite
%%%%%%%%%%%%%%%%%%%


\frontmatter
\titlehead{%  {\centering Seitenkopf}
  {Hochschule Heilbronn\\
   Fakultät für Informatik}}
\subject{Bachelorarbeit}
\title{\BaTitle\\[15mm]}
\subtitle{\normalsize{vorgelegt an der Hochschule Heilbronn, Fakultät für Informatik zum Abschluss eines Studiums im Studiengang Angewandte Informatik}}
\author{\BaAuthor\\
\normalsize{Matrikelnummer: \MatNr}}
\date{\normalsize{Eingereicht am: \BaDeadline}}
\publishers{
  \normalsize{Erstpr\"{u}fer: \BaSupervisorOne}\\
  \normalsize{Zweitpr\"{u}ferin: \BaSupervisorTwo}\\
}

%\uppertitleback{ }
%\lowertitleback{ }

\maketitle


%%%%%%%%%%%%%%%%%%%
%% abstract
%%%%%%%%%%%%%%%%%%%

\section*{Zusammenfassung}

TODO

\section*{Abstract}

TODO

\newpage
\chapter*{Danksagung}



%%%%%%%%%%%%%%%%%%%
%% Inhaltsverzeichnis
%%%%%%%%%%%%%%%%%%%
\tableofcontents



%%%%%%%%%%%%%%%%%%%
%% Main part of the thesis
%%%%%%%%%%%%%%%%%%%
\mainmatter

\chapter{Einführung}\label{ch:intro}

TODO. \cite{handbook}

\cite{handbook}

\chapter{Überblick Web-Application Security}

\section{OWASP: Open Web Application Security Project}
Das ``Open Web Application Security Project'' (OWASP) ist eine unabhängige, weltweite Community mit dem Ziel, die Bedeutung der Sicherheit von Webanwendungen »sichtbar zu machen«, Fachwissen zur Entwicklung und den Betrieb sicherer Webanwendungen zu verbreiten und frei zur Verfügung zu stellen.
OWASP ist mit keinem Technologieunternehmen verbunden, obwohl der Einsatz kommerzieller Sicherheitstechnologien unterstützt wird. Sämtliche OWASP-Instrumente, wie Dokumente, Videos, Slides, Podcasts etc. sind kollaborativ produziert worden und sind zudem kostenlos unter einer freien Lizenz verwendbar. Die OWASP Foundation ist eine Non-Profit-Organisation, die sich vier Grundwerten verschrieben hat:

\begin{itemize}
  \item Offenheit: Von den Finanzen bis zum Code ist alles radikal transparent.
  \item Innovation: OWASP fördert und unterstützt Innovationen und Experimente zur Lösung von Software-Sicherheitsherausforderungen.
  \item Globalität: Jeder auf der ganzen Welt kann sich an der OWASP-Community beteiligen.
  \item Integrität: OWASP ist eine ehrliche und aufrichtige, herstellerneutrale, globale Gemeinschaft.
\end{itemize}

Zudem gibt es einen Verhaltenskodex mit folgenden Prinzipien:
\begin{itemize}
  \item Führe alle beruflichen Tätigkeiten und Pflichten in Übereinstimmung mit allen anwendbaren Gesetzen und den höchsten ethischen Grundsätzen aus.
  \item Fördere die Umsetzung von Normen, Verfahren und Kontrollen für die Anwendungssicherheit und deren Einhaltung.
  \item Bewahre eine angemessenene Vertraulichkeit gegenüber geschützter oder anderweitig sensibler Informationen, die im Rahmen einer beruflichen Tätigkeit auftreten.
  \item Übernimm die berufliche Verantwortung mit Sorgfalt und Ehrlichkeit.
  \item Kommuniziere offen und ehrlich.
  \item Vermeide Aktivitäten, die einen Interessenskonflikt hervorrufen oder anderweitig den Ruf des Arbeitgebers, den Informationssicherheitsberuf oder die Vereinigung beeinträchtigen könnten.
  \item Bewahre und verstärke unser Objektivität und Unabhängigkeit.
  \item Weise unangemessenen Druck der von Seiten der Industrie oder anderen zurück.
  \item Verletze oder bestreite nicht absichtlich den Ruf von Kollegen, Kunden oder Arbeitgebern.
  \item Behandle jeden mit Respekt und Würde.
  \item Vermeide Beziehungen, die die Objektivität und Unabhängigkeit von OWASP  beeinträchtigen könnten.
\end{itemize}

\cite{OWASPabout}

\section{Sicherheitsrisiken: Die OWASP Top Ten}
TODO
\cite{OWASPtop10}

\subsection{A1: Injection}
Injection-Schwachstellen, wie beispielsweise SQL-, OS- oder LDAP-Injection, treten auf, wenn
nicht vertrauenswürdige Daten von einem Interpreter als Teil eines Kommandos oder einer
Abfrage verarbeitet werden. Ein Angreifer kann Eingabedaten dann so manipulieren, dass er nicht
vorgesehene Kommandos ausführen oder unautorisiert auf Daten zugreifen kann.

\subsection{A2: Fehler in der Authentifizierung}
Anwendungsfunktionen, die im Zusammenhang mit Authentifizierung und Sessionmanagement
stehen, werden häufig fehlerhaft implementiert. Dies erlaubt es Angreifern, Passwörter oder
Session-Token zu kompromittieren oder die entsprechenden Schwachstellen so auszunutzen,
dass sie die Identität anderer Benutzer vorübergehend oder dauerhaft annehmen können.

\subsection{A3: Verlust der Vertraulichkeit sensibler Daten}
Viele Anwendungen schützen sensible Daten, wie personenbezogene Informationen und Finanzoder
Gesundheitsdaten, nicht ausreichend. Angreifer können diese Daten auslesen oder
modifizieren und mit ihnen weitere Straftaten begehen (Kreditkartenbetrug, Identitätsdiebstahl
etc.). Vertrauliche Daten können kompromittiert werden, wenn sie nicht durch Maßnahmen, wie
Verschlüsselung gespeicherter Daten und verschlüsselte Datenübertragung, zusätzlich geschützt
werden. Besondere Vorsicht ist beim Datenaustausch mit Browsern angeraten.

\subsection{A4: XML External Entities}
Viele veraltete oder schlecht konfigurierte XML Prozessoren berücksichtigen Referenzen auf
externe Entitäten innerhalb von XML-Dokumenten. Dadurch können solche externen Entitäten
dazu eingesetzt werden, um mittels URI Datei-Handlern interne Dateien oder File-Shares offenzulegen
oder interne Port-Scans, Remote-Code-Executions oder Denial-of-Service Angriffe
auszuführen.

\subsection{A5: Fehler in der Zugriffskontrolle}
Häufig werden die Zugriffsrechte für authentifizierte Nutzer nicht korrekt um- bzw. durchgesetzt.
Angreifer können entsprechende Schwachstellen ausnutzen, um auf Funktionen oder Daten
zuzugreifen, für die sie keine Zugriffsberechtigung haben. Dies kann Zugriffe auf Accounts
anderer Nutzer sowie auf vertrauliche Daten oder aber die Manipulation von Nutzerdaten,
Zugriffsrechten etc. zur Folge haben.

\subsection{A6: Sicherheitsrelevante Fehlkonfiguration}
Fehlkonfigurationen von Sicherheitseinstellungen sind das am häufigsten auftretende Problem.
Ursachen sind unsichere Standardkonfigurationen, unvollständige oder ad-hoc durchgeführte
Konfigurationen, ungeschützte Cloud-Speicher, fehlkonfigurierte HTTP-Header und Fehlerausgaben,
die vertrauliche Daten enthalten. Betriebssysteme, Frameworks, Bibliotheken und Anwendungen
müssen sicher konfiguriert werden und zeitnah Patches und Updates erhalten.

\subsection{A7: Cross-Site-Scripting (XSS)}
XSS tritt auf, wenn Anwendungen nicht vertrauenswürdige Daten entgegennehmen und ohne
Validierung oder Umkodierung an einen Webbrowser senden. XSS tritt auch auf, wenn eine
Anwendung HTML- oder JavaScript-Code auf Basis von Nutzereingaben erzeugt. XSS erlaubt es
einem Angreifer, Scriptcode im Browser eines Opfers auszuführen und so Benutzersitzungen zu
übernehmen, Seiteninhalte verändert anzuzeigen oder den Benutzer auf bösartige Seiten
umzuleiten.

\subsection{A8: Unsichere Deserialisierung}
Unsichere, weil unzureichend geprüfte Deserialisierungen können zu Schwachstellen in der
Remote-Code-Execution führen. Aber auch wenn das nicht der Fall ist, können
Deserialisierungsfehler Angriffsmuster wie Replay-Angriffe, Injections und Erschleichung
erweiterter Zugriffsrechte ermöglichen.

\subsection{A9: Nutzung von Komponenten mit bekannten Schwachstellen}
Komponenten wie Bibliotheken, Frameworks etc. werden mit den Berechtigungen der zugehörigen
Anwendung ausgeführt. Wird eine verwundbare Komponente ausgenutzt, kann ein solcher
Angriff von Datenverlusten bis hin zu einer Übernahme des Systems führen. Applikationen und
APIs, die Komponenten mit bekannten Schwachstellen einsetzen, können Schutzmaßnahmen
unterlaufen und so Angriffe mit schwerwiegenden Auswirkungen verursachen.

\subsection{A10: Unzureichendes Logging und Monitoring}
Unzureichendes Logging und Monitoring führt zusammen mit fehlender oder uneffektiver
Reaktion auf Vorfälle zu andauernden oder wiederholten Angriffen. Auch können Angreifer
dadurch in Netzwerken weiter vordringen und Daten entwenden, verändern oder zerstören.
Viele Studien zeigen, dass die Zeit bis zur Aufdeckung eines Angriffs bei ca. 200 Tagen
liegt sowie typischerweise durch Dritte entdeckt wird und nicht durch interne
Überwachungs- und Kontrollmaßnahmen.\\
\\
\cite{OWASPtop10}

\section{Sicherheitstechnologien}
\subsection{White-Box testing}
TODO

\subsection{Black-Box testing}
TODO

\subsection{Fuzzing}
TODO

\subsection{Firewalls}
TODO

\subsection{Sonstige}
TODO

\section{Rechtliche Aspekte}
\subsection{Hacker/Cracker <-> Pentester}

TODO







\chapter{Auswahl der Tools: 1. Scanner}
\chapter{Auswahl der Tools: 1. Scanner}
\chapter{Auswahl der Tools: 1. Scanner}
\chapter{Auswahl der Tools: 2. Angriffstools}

\chapter{Lösung}

\begin{lstlisting}[label=lst:java,
				   language=java,
				   firstnumber=1,
				   caption=Beispiel für einen Quelltext]

public void foo() {
	// Kommentar
}
\end{lstlisting}

Lorem ipsum dolor sit amet, consectetur adipiscing elit. Ut vehicula felis lectus, nec aliquet arcu aliquam vitae. Quisque laoreet consequat ante, eget pretium quam hendrerit at. Pellentesque nec purus eget erat mattis varius. Nullam ut vulputate velit. Suspendisse in dui in eros iaculis tempus. Phasellus vel est arcu. Vestibulum ante ipsum primis in faucibus orci luctus et ultrices posuere cubilia Curae; Integer elementum, nulla eu faucibus dignissim, orci justo imperdiet lorem, luctus consectetur orci orci a nunc.

Praesent at nunc nec tortor viverra viverra. Morbi in feugiat lectus. Vestibulum iaculis ipsum at eros viverra volutpat in id ipsum. Donec condimentum, ligula viverra pharetra tincidunt, nunc dui malesuada nisi, vitae mollis lacus massa quis velit. Integer feugiat ipsum a volutpat scelerisque. Nulla facilisis augue nunc. Curabitur eget consectetur nulla. Integer accumsan sem non nisi tristique dictum.

Sed lacinia eu dolor sed congue. Ut dui orci, venenatis id interdum rhoncus, mattis elementum massa. Proin venenatis elementum purus ut rutrum. Phasellus sit amet enim porta, commodo mauris a, bibendum tortor. Nulla ut lobortis justo. Aenean auctor mi nec velit fermentum, quis ultricies odio viverra. Maecenas ultrices urna vel erat ornare, quis suscipit odio molestie. Donec vel dapibus orci, vel tincidunt orci.

Etiam vitae eros erat. Praesent nec accumsan turpis, et mollis eros. Praesent lacinia nulla at neque porta aliquam. Quisque elementum neque ac porta suscipit. Nulla volutpat luctus venenatis. Aliquam imperdiet suscipit pretium. Nunc feugiat lacinia aliquet. Mauris ut sapien nec risus porttitor bibendum. Aenean feugiat bibendum lectus, id mattis elit adipiscing at. Pellentesque interdum felis non risus iaculis euismod fermentum nec urna. Nullam lacinia suscipit erat ac ullamcorper. Sed vitae nulla posuere, posuere sem id, ultricies urna. Maecenas eros lorem, tempus non nulla vitae, ullamcorper egestas nibh. Vestibulum facilisis ante vel purus accumsan mattis. Donec molestie tempor eros, a gravida odio congue posuere.

Sed in tempus elit, sit amet suscipit quam. Ut suscipit dictum molestie. Etiam quis porta mauris. Cras dapibus sapien eget sem porta, ut congue sapien accumsan. Maecenas hendrerit lobortis mauris ut hendrerit. Suspendisse at aliquet est. Quisque eros est, scelerisque ac orci quis, placerat suscipit lorem. Phasellus rutrum enim non odio ullamcorper, sit amet auctor nulla fringilla. Nunc eleifend vulputate dui, a sollicitudin tellus venenatis non. Cras condimentum lorem at ultricies vestibulum. Vestibulum interdum lobortis commodo. Nullam rhoncus interdum massa, ut varius nisi scelerisque id. Nunc interdum quam in enim bibendum vulputate.


\chapter{Evaluierung}

Lorem ipsum dolor sit amet, consectetur adipiscing elit. Ut vehicula felis lectus, nec aliquet arcu aliquam vitae. Quisque laoreet consequat ante, eget pretium quam hendrerit at. Pellentesque nec purus eget erat mattis varius. Nullam ut vulputate velit. Suspendisse in dui in eros iaculis tempus. Phasellus vel est arcu. Vestibulum ante ipsum primis in faucibus orci luctus et ultrices posuere cubilia Curae; Integer elementum, nulla eu faucibus dignissim, orci justo imperdiet lorem, luctus consectetur orci orci a nunc.

Praesent at nunc nec tortor viverra viverra. Morbi in feugiat lectus. Vestibulum iaculis ipsum at eros viverra volutpat in id ipsum. Donec condimentum, ligula viverra pharetra tincidunt, nunc dui malesuada nisi, vitae mollis lacus massa quis velit. Integer feugiat ipsum a volutpat scelerisque. Nulla facilisis augue nunc. Curabitur eget consectetur nulla. Integer accumsan sem non nisi tristique dictum.

Sed lacinia eu dolor sed congue. Ut dui orci, venenatis id interdum rhoncus, mattis elementum massa. Proin venenatis elementum purus ut rutrum. Phasellus sit amet enim porta, commodo mauris a, bibendum tortor. Nulla ut lobortis justo. Aenean auctor mi nec velit fermentum, quis ultricies odio viverra. Maecenas ultrices urna vel erat ornare, quis suscipit odio molestie. Donec vel dapibus orci, vel tincidunt orci.

Etiam vitae eros erat. Praesent nec accumsan turpis, et mollis eros. Praesent lacinia nulla at neque porta aliquam. Quisque elementum neque ac porta suscipit. Nulla volutpat luctus venenatis. Aliquam imperdiet suscipit pretium. Nunc feugiat lacinia aliquet. Mauris ut sapien nec risus porttitor bibendum. Aenean feugiat bibendum lectus, id mattis elit adipiscing at. Pellentesque interdum felis non risus iaculis euismod fermentum nec urna. Nullam lacinia suscipit erat ac ullamcorper. Sed vitae nulla posuere, posuere sem id, ultricies urna. Maecenas eros lorem, tempus non nulla vitae, ullamcorper egestas nibh. Vestibulum facilisis ante vel purus accumsan mattis. Donec molestie tempor eros, a gravida odio congue posuere.

Sed in tempus elit, sit amet suscipit quam. Ut suscipit dictum molestie. Etiam quis porta mauris. Cras dapibus sapien eget sem porta, ut congue sapien accumsan. Maecenas hendrerit lobortis mauris ut hendrerit. Suspendisse at aliquet est. Quisque eros est, scelerisque ac orci quis, placerat suscipit lorem. Phasellus rutrum enim non odio ullamcorper, sit amet auctor nulla fringilla. Nunc eleifend vulputate dui, a sollicitudin tellus venenatis non. Cras condimentum lorem at ultricies vestibulum. Vestibulum interdum lobortis commodo. Nullam rhoncus interdum massa, ut varius nisi scelerisque id. Nunc interdum quam in enim bibendum vulputate.

\chapter{Zusammenfassung}


\backmatter
%%%%%%%%%%%%%%%%%%%
%% create figure list
%%%%%%%%%%%%%%%%%%%

\listoffigures
\addcontentsline{toc}{chapter}{Verzeichnisse}

%%%%%%%%%%%%%%%%%%%
%% create tables list
%%%%%%%%%%%%%%%%%%%
\listoftables

%%%%%%%%%%%%%%%%%%%
%% create listings list
%%%%%%%%%%%%%%%%%%%
%\lstlistoflistings
%\addcontentsline{toc}{chapter}{Listings}

\cleardoublepage
\phantomsection
\addcontentsline{toc}{chapter}{Literatur}
\printbibliography

%%%%%%%%%%%%%%%%%%%
%% declaration on oath
%%%%%%%%%%%%%%%%%%%

\addchap{Eidesstattliche Erklärung}

Hiermit versichere ich, dass ich die vorgelegte Bachelorarbeit selbstständig verfasst und noch nicht anderweitig zu Prüfungszwecken vorgelegt habe. Alle benutzten Quellen und Hilfsmittel sind angegeben, wörtliche und sinngemäße Zitate wurden als solche gekennzeichnet.

\vspace{20pt}
\begin{flushright}
$\overline{~~~~~~~~~~~~~~~~~\mbox{\BaAuthor, am \today}~~~~~~~~~~~~~~~~~}$
\end{flushright}

\addchap{Zustimmung zur Plagiatsüberprüfung}

Hiermit willige ich ein, dass zum Zwecke der Überprüfung auf Plagiate meine vorgelegte Arbeit in digitaler Form an PlagScan (www.plagscan.com) übermittelt und diese vorübergehend (max. 5~Jahre) in der von PlagScan geführten Datenbank gespeichert wird sowie persönliche Daten, die Teil dieser Arbeit sind, dort hinterlegt werden.

\begin{small}
Die Einwilligung ist freiwillig. Ohne diese Einwilligung kann unter Entfernung aller persönlichen Angaben und Wahrung der urheberrechtlichen Vorgaben die Plagiatsüberprüfung nicht verhindert werden. Die Einwilligung zur Speicherung und Verwendung der persönlichen Daten kann jederzeit durch Erklärung gegenüber der Fakultät widerrufen werden.
\end{small}

\vspace{20pt}
\begin{flushright}
$\overline{~~~~~~~~~~~~~~~~~\mbox{\BaAuthor, am \today}~~~~~~~~~~~~~~~~~}$
\end{flushright}

\end{document}
