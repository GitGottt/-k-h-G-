\documentclass[12pt,oneside,a4paper,parskip]{scrbook}
\usepackage[utf8]{inputenc}
\usepackage{csquotes}
\usepackage[ngerman]{babel}
\usepackage{floatflt}
\usepackage{subfigure}
\usepackage[pdftex]{graphicx}
\usepackage{wrapfig}
\usepackage[hidelinks]{hyperref}
\usepackage{color}
\usepackage{amssymb}
\usepackage{textcomp}
\usepackage{nicefrac}
\usepackage{scrhack}
\usepackage{pdfpages}
\usepackage{float}
\usepackage{pdflscape}
\usepackage{subfigure}
\usepackage{pdfpages}
\usepackage[verbose]{placeins}
\usepackage[headsepline,plainfootsepline]{scrlayer-scrpage}
\usepackage{listings}
%\usepackage{xcolor}
\usepackage[table]{xcolor}
%\usepackage{color}
\usepackage{caption}
\usepackage{subfigure}
\usepackage{epstopdf}
\usepackage{longtable}
\usepackage{setspace}
\usepackage{booktabs}
\usepackage[style=numeric,sorting=none,backend=bibtex]{biblatex}
\bibliography{sources}


%%%%%%%%%%%%%%%%%%%
%% definitions
%%%%%%%%%%%%%%%%%%%
\def\BaAuthor{Henning Janning}
\def\BaTitle{Evaluation von Web Application Vulnerability Scannern}
\def\BaSupervisorOne{Prof. Andreas Mayer}
\def\BaSupervisorTwo{Susanne Steuer (M.Sc.) }
\def\BaDeadline{05.04.2019}
\def\MatNr{192972}

\hypersetup{
pdfauthor={\BaAuthor},
pdftitle={\BaTitle},
pdfsubject={Subject},
pdfkeywords={Keywords}
}

%%%%%%%%%%%%%%%%%%%
%% configs to include
%%%%%%%%%%%%%%%%%%%
\colorlet{punct}{red!60!black}
\definecolor{background}{HTML}{EEEEEE}
\definecolor{delim}{RGB}{20,105,176}
\colorlet{numb}{magenta!60!black}

\definecolor{gray}{rgb}{0.4,0.4,0.4}
\definecolor{darkblue}{rgb}{0.0,0.0,0.6}
\definecolor{cyan}{rgb}{0.0,0.6,0.6}

\definecolor{pblue}{rgb}{0.13,0.13,1}
\definecolor{pgreen}{rgb}{0,0.5,0}
\definecolor{pred}{rgb}{0.9,0,0}
\definecolor{pgrey}{rgb}{0.46,0.45,0.48}

\lstset{
  basicstyle=\ttfamily,
  columns=fullflexible,
  showstringspaces=false,
  commentstyle=\color{gray}\upshape
  linewidth=\textwidth
}

\lstdefinelanguage{json}{
    basicstyle=\normalfont\ttfamily,
    numbers=left,
    numberstyle=\scriptsize,
    stepnumber=1,
    numbersep=8pt,
    showstringspaces=false,
    breaklines=true,
    backgroundcolor=\color{background},
    literate=
     *{0}{{{\color{numb}0}}}{1}
      {1}{{{\color{numb}1}}}{1}
      {2}{{{\color{numb}2}}}{1}
      {3}{{{\color{numb}3}}}{1}
      {4}{{{\color{numb}4}}}{1}
      {5}{{{\color{numb}5}}}{1}
      {6}{{{\color{numb}6}}}{1}
      {7}{{{\color{numb}7}}}{1}
      {8}{{{\color{numb}8}}}{1}
      {9}{{{\color{numb}9}}}{1}
      {:}{{{\color{punct}{:}}}}{1}
      {,}{{{\color{punct}{,}}}}{1}
      {\{}{{{\color{delim}{\{}}}}{1}
      {\}}{{{\color{delim}{\}}}}}{1}
      {[}{{{\color{delim}{[}}}}{1}
      {]}{{{\color{delim}{]}}}}{1},
}

\lstset{language=xml,
  morestring=[b]",
  morestring=[s]{>}{<},
  morecomment=[s]{<?}{?>},
  stringstyle=\color{black},
  numbers=left,
  numberstyle=\scriptsize,
  stepnumber=1,
  numbersep=8pt,
  identifierstyle=\color{darkblue},
  keywordstyle=\color{cyan},
  backgroundcolor=\color{background},
  morekeywords={xmlns,version,type}% list your attributes here
}

\lstset{language=Java,
  showspaces=false,
  showtabs=false,
  tabsize=4,
  breaklines=true,
  keepspaces=true,
  numbers=left,
  numberstyle=\scriptsize,
  stepnumber=1,
  numbersep=8pt,
  showstringspaces=false,
  breakatwhitespace=true,
  commentstyle=\color{pgreen},
  keywordstyle=\color{pblue},
  stringstyle=\color{pred},
  basicstyle=\ttfamily,
  backgroundcolor=\color{background},
%  moredelim=[il][\textcolor{pgrey}]{$$},
%  moredelim=[is][\textcolor{pgrey}]{\%\%}{\%\%}
}




\begin{document}

%%%%%%%%%%%%%%%%%%%
%% Titelseite
%%%%%%%%%%%%%%%%%%%


\frontmatter
\titlehead{%  {\centering Seitenkopf}
  {Hochschule Heilbronn\\
   Fakultät für Informatik}}
\subject{Bachelorarbeit}
\title{\BaTitle\\[15mm]}
\subtitle{\normalsize{vorgelegt an der Hochschule Heilbronn, Fakultät für Informatik zum Abschluss eines Studiums im Studiengang Angewandte Informatik}}
\author{\BaAuthor\\
\normalsize{Matrikelnummer: \MatNr}}
\date{\normalsize{Eingereicht am: \BaDeadline}}
\publishers{
  \normalsize{Erstpr\"{u}fer: \BaSupervisorOne}\\
  \normalsize{Zweitpr\"{u}ferin: \BaSupervisorTwo}\\
}

%\uppertitleback{ }
%\lowertitleback{ }

\maketitle


%%%%%%%%%%%%%%%%%%%
%% abstract
%%%%%%%%%%%%%%%%%%%

\section*{Zusammenfassung}

In dieser Arbeit werden aktuelle Web Vulnerability Scanner (WVS) auf ihren Umfang und ihre Tauglichkeit überprüft und verglichen. Jeder WVS wird an verschiedenen Web-Seiten getestet, die absichtlich eingebaute Schwachstellen haben, wie z.B. Juice-Shop oder Damn Vulnerable Web Application. Kriterien für die Bewertung der Tools sind die Anzahl der gefundenen Schwachstellen, Anzahl der False Positives und die daraus resultierende Trefferquote. Zudem fließen subjektive Eindrücke wie Handhabung oder intuitive Bedienung in die Bewertung mit ein.

\section*{Abstract}
TODO


\setcounter{secnumdepth}{3}
\setcounter{tocdepth}{3}
\tableofcontents

\listoffigures
\addcontentsline{toc}{chapter}{Abbildungsverzeichnis}

\listoftables
\addcontentsline{toc}{chapter}{Tabellenverzeichnis}

\mainmatter

\chapter{Einführung}\label{ch:intro}
In den letzten zwei Jahrzehnten hat sich das World Wide Web von einem reinen Informationsspeicher in eine Plattform mit hochfunktionalen Anwendungen verwandelt, die nicht nur sensible Daten verarbeiten sondern auch  Aktionen durchführen, die einflussreiche Auswirkungen auf die reale Welt haben.
Mit jeder Weiterentwicklung bringen Webanwendungen neue Sicherheitslücken mit sich und so verändern sich auch die Art und die Anzahl der am häufigsten auftretenden Fehler. Es gibt Angriffe auf Schwachstellen, die bei der Entwicklung der Webanwendungen noch nicht bekannt waren und daher nicht berücksichtigt wurden, andere Attacken haben an Bedeutung verloren, da das Bewusstsein für sie gestiegen ist oder aufgrund von Verbesserungen der Web-Browser Software. Neue Technologien bergen jedoch auch immer das Risiko von neuen Sicherheitslücken und so ist ``...in gewissem Maße die Sicherheit von Webanwendungen heute das bedeutendste Schlachtfeld zwischen Angreifern und solchen, die Daten schützen und Computerressourcen verteidigen müssen, und dies wird wahrscheinlich auf absehbare Zeit so bleiben.'' \cite{handbook}

\begin{figure}[htb!]
  \centering
    \includegraphics[width=1\textwidth]{Images/VulnByYear}
  \caption[Gefundene Schwachstellen pro Jahr laut CVE Details]{Gefundene Schwachstellen pro Jahr laut CVE Details \cite{cve}}
\end{figure}

Mit der Anzahl der Webanwendungen steigt auch der Einfluss des World Wide Webs auf alle Lebensbereiche, sei es beim Einkaufen im Online-Shop einschließlich diverser Bezahlsysteme, dem Nachrichtenaustausch über Social-Media-Kanäle oder nur zur Informationsgewinnung auf Nachrichtenseiten - die Gesellschaft ist mehr denn je abhängig von der immer größer werdenden Anzahl an verschiedenen Webanwendungen im Internet.
Fortschreitende Digitalisierung und weltweite Vernetzung verursachen jedoch auch immer mehr Sicherheitslücken: laut CVE Details hat sich die Anzahl der gefundenen Schwachstellen in den letzten zwei Jahren mehr als verdoppelt (siehe Abb. 1.1).


Zunehmende Cyberangriffe von kriminellen Hackern aber auch von politisch oder ideologisch motivierten Angreifern sind die Folge. Aktuelle Beispiele sind der Angriff auf die Münchner Firma Krauss Maffai im Dezember 2018, der die Produktion des Maschinen-bauunternehmens für mehrere Tage lahmlegte oder die Attacke auf das Datennetz des Deutschen Bundestags im Oktober 2018. Das Datenleck „Collection \texttt{\#}1“, das im Januar 2019 auftauchte, ist das Ergebnis einer Vielzahl von Cyberattacken, es enthält über 2,6 Milliarden Datensätze mit Zugangsdaten und Passwörtern für hunderte Webseiten.

Für die betroffenen Firmen ist der Schaden enorm: Laut einer Studie des Digitalverbands Bitkom lag der durch Cyberattacken verursachte wirtschaftliche Gesamtschaden für Industrieunternehmen in Deutschland innerhalb der letzten zwei Jahre bei über 43 Milliarden Euro \cite{Bitkom}. Aus der gleichen Studie geht hervor, dass unentdeckte Sicherheitslücken das größte Risiko darstellen (siehe Abb. 1.2)\\

\begin{figure}[htb!]
  \centering
   \includegraphics[width=1\textwidth]{Images/Bitkom}
  \caption[Bitkom Studie - Bedrohungsszenarien]{Bitkom Studie - Bedrohungsszenarien \cite{Bitkom}}
\end{figure}
Es gibt mehrere Ansätze, diesem Risiko zu begegnen: Grundsätzlich sollte der Entwickler einer Webseite von Beginn an eine Programmierung anstreben, die bereits bekannte Sicherheitslücken vermeidet und so potentiellen Angreifern möglichst wenig Angriffsfläche bietet. Hier bietet das Bundesamt für Sicherheit in der Informationstechnik (BSI) mit seinem ``Leitfaden zur Entwicklung sicherer Webanwendungen''  \cite{BSI} Hilfestellung.\\
Fehler in der Programmierung lassen sich jedoch nicht immer ausschließen, zudem ist auch ein Schutz gegen unentdeckte Sicherheitslücken vonnöten.

Neben dem Einsatz von Web Application Firewalls ("Web Shields"), die den Datenstrom zwischen Browser und Webapplikation überwachen, ist die Verwendung von Web Application Vulnerability Scannern (WVS) im Rahmen von Penetration Tests ein wesentlicher Bestandteil von Sicherheitskonzepten in diesem Bereich.

WVS überprüfen Webanwendungen automatisiert auf Schwachstellen und unterstützen dadurch den Penetration Tester beim Aufspüren von Sicherheitslücken.

Die vorliegende Arbeit macht sich die Evaluation von WVS zur Aufgabe, da die wenigen bisherigen Ausarbeitungen zu diesem Thema veraltet sind und teilweise andere Ansätze verfolgen:

\begin{itemize}
  \item
  Holm \cite{Holm} vergleicht in seiner Studie aus dem Jahr 2011 lediglich kommerzielle Scanner und verzichtet auf Bewertungskategorien wie Bedienung und Reporting.
  \item
  Wundram \cite{Wundram}\cite{Wundram2} behandelt das Thema zweimal, in seinen Vergleichen von 2011 und 2012 finden sich veraltete Programme wie Watobo oder das inzwischen nicht mehr lauffähige W3af.
\end{itemize}

Zusätzliche Motivation ist die große Anzahl und Vielfalt der Scanner: Das Open Web Application Security Project (OWASP) listet allein eine Sammlung von 50 verschiedenen WVS auf \cite{OWASPtools}, es gibt freie, Open Source und kommerzielle WVS jeweils für verschiedene Plattformen, reine Terminal-Anwendungen und Programme mit grafischer Benutzeroberfläche.
Das Ziel dieser Arbeit ist es, aus dieser bunten Mischung diejenigen WVS herauszufiltern, für die eine nähere Betrachtung lohnenswert erscheint und diese im Hinblick auf folgende Fragestellungen zu evaluieren:

\textbf{1. Wieviele Schwachstellen werden von den WVS gefunden?}

\textbf{2. Wie unterscheiden sich die WVS in den Kategorien Bedienung, Reporting und Scan-Geschwindigkeit?}

\textbf{3. Welcher WVS schneidet insgesamt am besten ab?}

\textbf{4. Gibt es Unterschiede zwischen Open Source- und kommerziellen WVS?}

Die Evaluation soll mit Hilfe eines Punktesystems erfolgen und schließlich in einem Ranking resultieren, das die Tauglichkeit und Qualität der getesteten WVS anhand ihrer Platzierungen aufzeigt.

Die Arbeit ist in sechs Kapitel aufgeteilt:

\textbf{1. Einführung:} In der Einführung wird der Leser an das Thema herangeführt, die Motivation und die Zielsetzung  sowie die zentralen Fragestellungen und der Aufbau der Arbeit werden erläutert.

\textbf{2. Grundlagen:} Das zweite Kapitel vermittelt Grundlagen über Webanwendungen, Schwachstellen,  Sicherheitsmaßnahmen wie Web Application Firewalls und Penetrationtesting sowie die Arbeistweise von WVS.

\textbf{3. Methodik:} Hier wird der Testaufbau beschrieben einschließlich der Auswahlkriterien für die WVS und des Punktesystems für die Bewertung. Nicht berücksichtigte WVS werden beschrieben sowie die verwundbaren Webanwendungen, die für die Tests verwendet wurden.

\textbf{4. Evaluation:} Dieses Kapitel beinhaltet die Ergebnisse der Evaluation, zum einen die Anzahl der gefundenen Schwachstellen per WVS, zum anderen eine Beschreibung der WVS einschließlich der Bewertung in den Kategorien Bedienung, Reporting und Scangeschwindigkeit.

\textbf{5. Diskussion:} In diesem Kapitel werden die Ergebnisse der Evaluation durchleuchtet und erörtert.

\textbf{6. Fazit und Ausblick:} Zum Schluss wird die Studie nochmals zusammengefasst und auf mögliche zukünftige Arbeiten verwiesen.\\

Dieser Arbeit wurde eine CD-ROM beigelegt, die sämtliche Berichte aller durchgführten Scans enthält.

\chapter{Grundlagen}
  \section{Webanwendung}
  Zu Beginn sollte der Begriff ``Webanwendung'' geklärt werden.
  Eine Webanwendung muss nicht zwingend über das World Wide Web erreichbar sein, auch in vielen Unternehmen kommen Webanwendungen zum Einsatz. Entscheidend dafür, dass sich eine Anwendung als Webanwendung bezeichnen lässt, ist stattdessen einzig der Einsatz von Webtechnologien. Hierüber gelangen wir zu folgender Begriffsdefinition \cite{Rohr}:

  \begin{quote}Eine Webanwendung ist eine Client-Server-Anwendung, die auf Webtechnologien (HTTP, HTML etc.) basiert.\end{quote}

  Eine Webanwendung wird über einen Webbrowser aufgerufen, der den serverseitig bereitgestellten HTML-, Java-Script oder CSS-Code interpretiert und darstellt. Danben kann auch über ein Skript oder von einer Kommandozeile aus auf Webanwendungen zugegriffen werden, man spricht hier von einem User Agent oder Client.
  Zur Kommunikation zwischen Browser (also Client) und Server wird das HTTP-Protokoll verwendet oder das darauf aufsetzende HTTPS-Protokoll.
  Serverseitig werden Webanwendungen auf Web- und Applikationsservern oder Laufzeitumgebungen ausgeführt, die dann wiederum auf Hintergrundsysteme wie Datenbanken zugreifen können.

  Daraus ergibt sich eine sogenannte 3-Tier-Architektur (dreischichtige Architektur, siehe Abb. 2.1).
  \begin{figure}[H]
    \centering
     \includegraphics[width=0.7\textwidth]{Images/3_Tier_Architektur}
    \caption[3-Tier-Architektur einer Webanwendung]{3-Tier-Architektur einer Webanwendung \cite{Rohr}}
  \end{figure}

  Moderne Webanwendungen lassen sich jedoch - insbesondere im Enterprise-Umfeld - nicht als einzelne Anwendungen sehen, sondern als Zusammenschluss verschiedener eigenständiger Dienste wie REST- oder Microservices zu einer Plattform, wie z.B. einem Onlineshop (siehe Abb. 2.2).
  \begin{figure}[H]
    \centering
     \includegraphics[width=1\textwidth]{Images/Enterprise}
    \caption[Enterprise Webanwendung auf Basis einer Microservice-Architektur]{Enterprise Webanwendung auf Basis einer Microservice-Architektur \cite{Rohr}}
  \end{figure}

  Im Zuge der Weiterentwicklung von Webanwendungen werden immer mehr Aspekte der Benutzerschnittstelle client-seitig, vor allem über JavaScript-Code umgesetzt, der im Hintergrund serverseitige Webdienste aufruft.
  Diese Verlagerung von Anwendungslogik vom Server auf den Client findet ihren Höhepunkt in sogenannten Single Page Applications (SPAs), die nur noch aus einer einzigen HTML-Seite mit sehr viel JavaScript-Code bestehen, der im Hintergrund dynamisch mit serverseitigen REST-Services kommuniziert und die Anzeige von Seiteninhalten steuert. Ein bekanntes Beispiel für solch eine Single Page Application ist Google Mail.
  \cite{Rohr}
  \newpage

 \section{Schwachstellen}
  Das Bundesamt für Sicherheit in der Informationstechnik (BSI) definiert eine Schwachstelle wie folgt:

  \begin{quote}``Eine Schwachstelle (englisch ``vulnerability'') ist ein sicherheitsrelevanter Fehler eines IT-Systems oder einer Institution. Ursachen können in der Konzeption, den verwendeten Algorithmen, der Implementation, der Konfiguration, dem Betrieb sowie der Organisation liegen. Eine Schwachstelle kann dazu führen, dass eine Bedrohung wirksam wird und eine Institution oder ein System geschädigt wird. Durch eine Schwachstelle wird ein Objekt (eine Institution oder ein System) anfällig für Bedrohungen.'' \cite{BSI2}
  \end{quote}

  Viele Schwachstellen werden nicht öffentlich bekannt, da ein Hersteller sie im Idealfall
  selbst entdeckt und behebt, bevor ein Dritter sie finden und insbesondere ein Angreifer
  sie ausnutzen kann.
  Wird eine sicherheitskritische Schwachstelle jedoch nicht durch den Hersteller, sondern
  durch einen Dritten gefunden, gibt es unterschiedliche Möglichkeiten, wie ihr Lebenszyklus und die daraus resultierende Gefährdung für den Nutzer verlaufen können.
  Wichtig sind dabei Rolle und Selbstverständnis des Entdeckers. Neben Personen, die
  sich beruflich mit IT-Sicherheit beschäftigen, kann beispielsweise auch akademische
  Forschung zur Entdeckung von Schwachstellen beitragen. Es kann sich um gezielte Suche nach Schwachstellen handeln, aber auch Zufallsfunde sind möglich.
  Der Lebenszyklus hängt daher stark von Verhalten und Motivation des Entdeckers sowie von der Reaktion des Herstellers ab \cite{BSI3}:
  \begin{itemize}
    \item Ist dem Entdecker daran gelegen, die IT-Sicherheit zu verbessern, oder möchte er die
    Schwachstelle für eigene, möglicherweise kriminelle Zwecke nutzen?
    \item Wie viele Informationen über die Schwachstelle werden öffentlich bekannt?
    \item Schätzt der Hersteller die entstehende Gefährdung richtig ein und kann er schnell genug Abhilfe
  schaffen?
  \end{itemize}
  Um die Diskussion über Schwachstellen zwischen Entdeckern, Herstellern und weiteren
  Beteiligten zu systematisieren, wurde zur einheitlichen Benennung von öffentlich bekannten Schwachstellen und zur Sammlung der darüber verfügbaren Informationen
  mit den Common Vulnerabilities and Exposures (CVE) ein herstellerübergreifender Industriestandard geschaffen. Mit diesem Standard wird sichergestellt, dass alle Beteiligten während des gesamten Lebenszyklus tatsächlich jeweils dieselbe Schwachstelle meinen. Zudem werden statistische Auswertungen ermöglicht.

  Grundsätzlich basiert der Lebenszyklus einer durch Dritte entdeckten Schwachstelle auf dem
  folgenden Schema \cite{BSI3}:

  1. Die Schwachstelle wird durch einen Dritten entdeckt und untersucht.

  2. Der Hersteller erlangt auf einem der folgenden Wege Kenntnis von der Schwachstelle:
  \begin{itemize}
    \item Der Entdecker veröffentlicht sämtliche Informationen über die Schwachstelle (\glqq Full Disclosure\grqq).
    \item Der Entdecker informiert den Hersteller direkt und verzichtet zunächst auf eine detaillierte
    Veröffentlichung (\glqq Coordinated Disclosure\grqq).
    \item Die Schwachstelle wird für Angriffe ausgenutzt und diese werden entdeckt (\glqq Zero-Day-Exploit\grqq).
    \item Der Hersteller wird indirekt über einen sogenannten Schwachstellen-Broker über die Schwachstelle informiert.
  \end{itemize}
  3. Der Hersteller beginnt mit der Entwicklung eines Patches (von engl. Flicken), einer Nachbesserung der Software, mit der die Schwachstelle behoben wird.

  4. Der Hersteller veröffentlicht eine Schwachstellenwarnung, ein sogenanntes Advisory. Dieses
  enthält üblicherweise allgemeine Informationen zur Schwachstelle, eine Bewertung der
  entstehenden Gefährdung sowie mögliche vorläufige Gegenmaßnahmen (sogenannte
  Workarounds oder Mitigations). Ein Advisory wird vor allem dann schon vor Verfügbarkeit
  des Patches veröffentlicht, wenn auch die Öffentlichkeit bereits Kenntnis von der Schwachstelle hat und die Gefährdung hoch ist. Häufig erscheinen Advisory und Patch jedoch zeitgleich.

  5. Der Hersteller stellt zusammen mit einer entsprechenden Beschreibung (Bulletin) einen
  Patch zur Behebung der Schwachstelle bereit. Durch Analyse des Patches kann ein Angreifer
  unter Umständen so viele Details über die Schwachstelle herausfinden, dass er sie ausnutzen
  kann. Daher steigt mit der Veröffentlichung des Patches die Gefährdung für ungepatchte
  Systeme an.

  6. Der Benutzer der betroffenen Software installiert den Patch, schließt dadurch die Schwachstelle und ist erst dann vor ihrer Ausnutzung geschützt. Dies unterstreicht die enorme Bedeutung eines effektiven Patchmanagements.
  \begin{figure}[hbt!]
    \centering
     \includegraphics[width=1\textwidth]{Images/Lebenszyklus}
    \caption[Lebenszyklus einer Schwachstelle]{Lebenszyklus einer Schwachstelle \cite{BSI3}}
  \end{figure}

  Abweichungen von diesem Schema sind möglich, der Ablauf kann sich dynamisch ändern. So
  kann ein Hersteller beispielsweise eine Schwachstelle als unkritisch einschätzen und auf die
  Entwicklung eines Patches zunächst verzichten. Wird zu einem späteren Zeitpunkt doch eine
  Ausnutzungsmöglichkeit bekannt, so kann der Hersteller dadurch zu einer entsprechenden
  schnellen Reaktion gezwungen sein. \cite{BSI3}

  \section{Web Application Security}
  Die Webanwendungssicherheit ist ein Teilgebiet der IT-Sicherheit und befasst sich vor allem mit dem Schutz von Assets einer Webanwendung. Unter Assets versteht man in diesem Fall all jene Bestandteile
  (Daten, Systeme und Funktionen) einer Webanwendung, für die sich ein Schutzbedarf hinsichtlich eines der
  drei primären Schutzziele ableiten lässt.
  Diese primären Schutzziele lauten:
    \begin{itemize}
    \item Vertraulichkeit
    \item Integrität
    \item Verfügbarkeit
  \end{itemize}

  Aus diesen primären Schutzzielen lassen sich auch sekundäre Schutzziele wie z.B. Authentizität oder Nicht-Abstreitbarkeit ableiten.

  Das Thema Webanwendungssicherheit bezieht sich auf alle Phasen des Lebenszyklus einer Webanwendung und stellt auch einen Bestandteil der Qualitätssicherung dar. Sie befasst sich sowohl mit der Abwehr von Bedrohungen und der Prävention von Schwachstellen als auch mit der Identifizierung und Behebung von Sicherheitslücken. \cite{BSI}

  \subsection{Bedrohungen und Risiken}
  Die schwerwiegendsten Angriffe auf Webanwendungen sind sicherlich diejenigen, die sensible Daten verfügbar machen oder uneingeschränkten Zugriff auf die Systeme ermöglichen, auf denen die Anwendung ausgeführt wird.
  Hierzu gehören Angriffe per XSS oder SQL-Injection, aber auch Angriffe, die Systemausfälle versursachen (Denial-of-Service Attacken) stellen für viele Organisationen ein sehr kritisches Ereignis dar.

  OWASP stellt in regelmäßigen Abständen seine OWASP Top Ten vor, eine Liste mit den 10 kritischsten Sicherheitsrisiken für Webanwendungen. Die aktuelle Version aus dem Jahr 2017 befindet sich im Anhang dieser Ausarbeitung.

  \subsection{Sicherheitsmaßnahmen}
  Das BSI definiert den Begriff wie folgt:
    \begin{quote} ``Als Sicherheitsmaßnahmen werden alle Aktionen bezeichnet, die dazu dienen, Sicherheitsrisiken zu steuern und diesen entgegenzuwirken. Dies schließt sowohl organisatorische, als auch personelle, technische oder infrastrukturelle Sicherheitsmaßnahmen ein. Synonym werden auch die Begriffe Sicherheitsvorkehrung oder Schutzmaßnahme benutzt.'' \cite{BSI2}\end{quote}

 Zu den Maßnahmen, die bereits während der Entwicklung ergriffen werden sollten, gehören umfangreiche Security Tests. Nach Inbetriebnahme der Webanwendung kann mit Hilfe von Penetration Tests nach Sicherheitslücken gesucht werden, zusätzlichen Schutz bieten Web Application Firewalls.

  \subsubsection{Security Tests}

  Bevor eine Webanwendung ausgeliefert wird, sollte sie im Zuge eines oder mehrerer Security Assessments  auf Sicherheitsmängel getestet werden.
  Ein Security Assessment soll dabei sicherstellen, dass die definierten Sicherheitsanforderungen in der
  Implementierungsphase korrekt umgesetzt worden sind. Zusätzlich soll in diesem Rahmen auch die
  Effektivität der gesetzten Sicherheitsanforderungen überprüft werden und so mögliche Unzulänglichkeiten
  in der Spezifikation selbst aufgedeckt werden. Das heißt, es soll sichergestellt werden, dass die spezifizierten Sicherheitsanforderungen auch angemessen und wirksam sind.
  Üblicherweise durchlaufen Security Assessments die folgenden Phasen \cite{BSI}:
  \begin{itemize}
    \item Planung: Ein Security Assessment benötigt eine initiale Planung. Hier werden sämtliche
    Informationen, die für die Durchführung des Assessments notwendig sind, zusammengetragen.
    \item Durchführung: Die Hauptaufgabe dieser Phase ist das Suchen nach Schwachstellen und deren
    Bewertung.
    \item Auswertung: Hierbei werden die gefundenen Schwachstellen analysiert. Es werden die Ursachen
    und die Gegenmaßnahmen bestimmt und in einem Abschlussbericht zusammengefasst. Der
    Abschlussbericht wird an die Entwickler geleitet, die die identifizierten Schwachstellen beheben.
    Nach der Behebung werden die Schwachstellen erneut untersucht.
  \end{itemize}
  Die Überprüfung kann unter Verwendung mehrerer verschiedener Testverfahren erfolgen. Dabei ist zu
  berücksichtigen, dass unterschiedliche Testverfahren unterschiedliche Aufgaben erfüllen und
  unterschiedliche Ergebnisse liefern. Beispielsweise überprüfen Unit Tests die Korrektheit und die
  vollständige Abdeckung einer Funktionalität, während ein Code Review die korrekte Implementierung
  überprüft. Je nachdem was genau überprüft werden soll, können auch unterschiedliche Testrollen
  herangezogen werden. Ein Auditor kann beispielsweise die Einhaltung von Vorgaben überprüfen, während
  ein Penetrationstester die korrekte Implementierung überprüft.

  Folgende Testverfahren können bereits während der Implementierung durchgeführt werden \cite{BSI}:
  \begin{itemize}
    \item Security Test Cases:
    Ein Test Case beschreibt einen Softwaretest, in dem die laut den Sicherheitsanforderungen
    bestimmte spezifizierte Funktionalität (funktional oder nicht funktional) der Webanwendung auf
    ihre Korrektheit überprüft wird. Test Cases einer Webanwendung werden hierbei in einfacher
    Sprache beschrieben.
    Ein Test Case besteht immer zumindest aus einer eindeutigen ID, einer Beschreibung und einem
    erwarteten Ergebnis.
    Im Zuge des Tests Cases muss die Einhaltung des erwarteten Ergebnisses überprüft werden.
    Test Cases lassen sich in folgende Arten unterteilen:

    - Positivtest: Das Verhalten der Webanwendung wird mit gültigen Rahmenbedingungen und
    Eingaben überprüft.

    - Negativtest: Das Verhalten der Webanwendung wird mit ungültigen Rahmenbedingungen
    und Eingaben überprüft.

    Test Cases sollten priorisiert werden. Jedem Test Case wird ein je nach Kritikalität der zugehörigen
    Funktion ein Prioritätslevel (z.B. Normal, Hoch, Kritisch) zugewiesen und beim Testen
    berücksichtigt.
    Test Cases müssen im Rahmen der Entwicklung durch den Entwickler erstellt werden.
    \item Security Unit Tests:
    Unit Tests überprüfen Softwareeinheiten (Units) einer Webanwendung auf korrekte Funktionalität.
    In der Regel sind das Funktionen, Methoden oder Klassen. Units werden mit verschiedenen
    Parametern aufgerufen und es wird überprüft, ob die Ausgabe mit den Erwartungen
    übereinstimmt.
    Unit Tests müssen im Rahmen der Entwicklung durch die Entwickler erstellt und gepflegt werden.
    \item Design Review:
    Bei Design Reviews geht es darum, die Architektur der Anwendung auf hoher Ebene zu
    untersuchen. Dabei sollen konzeptionelle Fehler und Verwundbarkeiten aufgedeckt werden und
    die Umsetzung der Sicherheitsanforderungen und -mechanismen auf ihre Vollständigkeit hin
    untersucht werden. Wenn erhebliche Mängel im Design erkannt werden, muss die Konzeption und die Planung überarbeitet werden. Die im Design Review erkannten Mängel dienen hierbei als Ansatzpunkt.
    \item Code Reviews:
    Beim Code Review wird der Programmcode manuell überprüft. Hierbei ist zu beachten, dass die
    Durchsicht durch eine andere Person erfolgt (etwa innerhalb eines Peer Reviews) als derjenigen, die
    den Programmcode verfasst hat, da ansonsten Fehler leicht übersehen werden können. Typische
    Fehler, die entdeckt werden können, sind Abweichungen von Standards, Abweichungen gegenüber
    Anforderungen, Fehler im Design, Buffer Overflows etc. In erster Linie ist das bei sensitiven
    Funktionen wie beispielsweise Transaktionsmanagement, Authentisierung und Kryptographie
    wichtig. Die Ergebnisse des Code Reviews fließen zurück in den Implementierungsprozess.
    \item Statische Code Scanner:
    Mittels Statischer Code Analyse (SCA) können eine Vielzahl von Schwachstellen bereits innerhalb
    der Entwicklung identifiziert und so zeitnah behandelt werden. In der Praxis ist die Auswahl
    entsprechender Tools zur Durchführung von Sicherheitsscans sehr beschränkt. Weiterhin müssen
    diese Tools die eingesetzten Technologien unterstützen. Analysieren sie den Sourcecode, müssen
    diese zusätzlich die eingesetzten APIs kennen, da diese gewöhnlich nicht im Sourcecode vorliegen.
    Schließlich erzeugen SCA-Tools häufig eine sehr hohe Anzahl an False Positives, weshalb eine
    Verifikation der Scanergebnisse durch fachmännisches Personal erforderlich ist.
    \item Web Application Vulnerability Scanner: siehe 2.3
    \item Fuzz Testing:
    Fuzz Testing ist ein Testverfahren, welches automatisiert oder teilweise automatisiert durchgeführt
    wird. Dabei werden ungültige, unerwartete oder zufällige Werte generiert und als Eingabe an die
    Anwendung weitergegeben. Das Verhalten der Anwendung wird während des Tests auf
    Fehlverhalten überwacht. Das Prinzip ist simpel und führt dazu, dass Fehler entdeckt werden, die
    oft übersehen worden wären. Fuzz Testing kann nur einfache Programmfehler entdecken, die auf
    falsche Eingabewerte zurückzuführen sind, jedoch keine Designfehler etc. \cite{BSI}
  \end{itemize}

    \subsubsection{Penetration Testing}
    Nach Fertigstellung der Webanwendung kommen Penetration Tests zum Einsatz. Sie können als legaler und autorisierter Versuch definiert werden, Computersysteme anzugreifen, mit dem Ziel, diese sicherer zu machen.
    Der Prozess umfasst die Suche nach Schwachstellen sowie die Demonstration von Beispielangriffen, um zu zeigen, dass die Bedrohungen real sind.
    Ein ordnungsgemäßer Penetration Test resultiert immer in spezifischen Empfehlungen für das
    Beheben der während des Tests aufgetretenen Probleme.
    Insgesamt wird dieses Verfahren dazu verwendet, Computer und Netzwerke gegen zukünftige Angriffe abzusichern. Die allgemeine Idee besteht darin, Sicherheitslücken mithilfe der gleichen Tools und Techniken zu finden, die auch ein Angreifer benutzt. Die Lücken können so geschlossen werden, bevor ein realer Hacker sie ausnutzt. \cite{engebretson}

    Üblicherweise lassen sich Penetration Tests in mehrere Phasen unterteilen \cite{BSI}:
    \begin{itemize}
      \item Informationssammlung über das Zielsystem
      \item Scan nach angebotenen Diensten
      \item Identifikation des Systems und der Anwendungen
      \item Recherche nach Schwachstellen
      \item Ausnutzen der Schwachstellen
    \end{itemize}

    Penetration Tests können basierend auf unterschiedlichen Kriterien klassifiziert werden \cite{BSI}:
    \begin{itemize}
      \item Informationsbasis: Hier wird festgelegt, von welchem Wissensstand der Tester ausgeht.
      \item Aggressivität: Hier wird festgelegt, wie aggressiv der Angreifer vorgeht. z.B. passiv (gefundene
      Schwachstellen werden nicht ausgenutzt) oder aggressiv (gefundene Schwachstellen werden
      unabhängig von den Konsequenzen ausgenutzt).
      \item Umfang: Hier wird festgelegt, welche Systeme getestet werden.
      \item Vorgehensweise: Der Angreifer kann verdeckt (Penetration wird nicht direkt als Angriff erkannt)
      oder offensichtlich (Penetration erfolgt offensichtlich) vorgehen.
      \item Technik: Es muss definiert werden, welche Techniken eingesetzt werden. Erfolgt der Angriff z.B. nur
      über das Netzwerk oder auch über weitere Kommunikationsnetze (Telefon, physischer Zugriff etc.)?
      \item Ausgangspunkt: Hier wird festgelegt, ob der Penetration Test von innen oder von außen erfolgen
      soll. Man unterscheidet hier zwischen drei unterschiedlichen Ansätzen, dem Black-Box, Grey-Box und White- Box Testing:

    Beim Black-Box Testing befindet sich der Tester in der Rolle eines typischen Hackers von
    außen, der kein Wissen über die innere Arbeitsweise der Anwendung hat, weder Architektur noch
    Quellcode sind bekannt. Der Angreifer muss mit manuellen Methoden sowie speziellen Tools des Penetration Testings vertraut sein, um Schwachstellen zu lokalisieren und auszunutzen. Für die dynamische Analyse des anzugreifenden Netzwerks benötigt er Scanning Tools, die in der vorliegenden Arbeit evaluiert werden.

    Während der Black-Box-Tester ein System aus der Sicht eines Außenseiters untersucht, hat ein    Grey-Box Tester bereits Zugriff auf das System auf Benutzerebene, möglicherweise sogar mit erhöhten Berechtigungen eines Administrators. In der Regel liegt eine Dokumentation über Design und Architektur des Netzwerks vor, die internen Komponenten sind bekannt. Dies hat den Vorteil, dass die Sicherheit des Netzwerks gezielter und effizienter beurteilt werden kann, der Tester kann sich sofort auf die Systeme konzentrieren, die am wichtigsten sind oder ein besonders hohes Risiko haben. Zudem kann durch das interne Benutzerkonto ein Angriff innerhalb des abgesicherten Systems mit umfassendem Zugriff auf das Netzwerk simuliert werden.

    Der White-Box Tester hat schließlich uneingeschränkten Zugriff auf ein System, er besitzt alle nötigen Berechtigungen und kennt Netzwerkarchitektur und Design. Überdies hat er Zugang zum Quellcode, was es ihm erlaubt, statische Code-Analysen durchzuführen. Durch das Auffinden sowohl interner als auch externer Schwachstellen ist das White-Box Testing maximal effektiv, aber auch sehr aufwandsintensiv da der Tester sehr große Datenmengen untersuchen muss.
    \end{itemize}

    Penetration Tests können nicht nur auf technische Systeme, sondern auch auf die organisatorische oder
    personelle Infrastruktur in Form von Social Engineering erfolgen.
    Unabhängig von der Vorgehensweise haben Penetration Tests folgende Ziele \cite{BSI}:
    \begin{itemize}
      \item Erhöhung der Sicherheit der IT-Systeme
      \item Identifikation von Schwachstellen
      \item Bestätigung der Sicherheit der IT-Systeme von unabhängigen Dritten
      \item Erhöhung der Sicherheit der organisatorischen und personellen Infrastruktur
    \end{itemize}

    In jedem Fall liefern Penetration Tests die Sicherheitslücken zwischen Planung und Implementierung. Die
    getroffenen Maßnahmen, um die Lücken zu schließen, müssen in einem weiteren Schritt nicht nur auf ihre
    korrekte Umsetzung überprüft werden, sondern auf ihre Angemessenheit, wie effektiv sie das Risiko
    tatsächlich senken beziehungsweise beseitigen oder ob durch sie ein falsches Sicherheitsgefühl vermittelt
    wird. Üblicherweise werden Penetration Tests erst dann durchgeführt, wenn die Webanwendung mit allen
    Komponenten fertiggestellt wird. \cite{BSI}
    \subsubsection{Web Application Firewalls (WAFs)}
    Zusätzlichen Schutz von Angriffen auf Webapplikationen bieten WAFs, die den Verkehr zwischen Clients und Webservern auf Anwendungsebene überprüfen. Sie sind in der Lage, HTTP-Traffic zu filtern und gegebenenfalls zu blockieren, um die Webanwendung zu schützen.
    Hierzu untersucht eine WAF alle eingehenden Anfragen und die ausgehenden Antworten des Webservers.
    Erkennt sie dabei gefährliche Muster, verhindert sie die weitere Kommunikation mit dem Client.

    Eine WAF kann zentralisiert hinter der Netzerk Firewall und vor dem Webserver positioniert oder Host-basiert als Software-Lösung direkt auf dem Webserver installiert werden.
    Häufig wird der sogenannte Reverse-Proxy-Modus verwendet, bei dem der Proxy sich zwischen Webserver und Firewall befindet und Zugriffe im Namen des Clients durchführt. Im zweiten Schritt werden die Anfragen an den eigentlichen Webserver analysiert und die Websessions bei Bedarf terminiert.

    Zu den üblichen Angriffen, die eine WAF verhindern kann, zählen Cross-Site-Scripting, SQL-Injection,
    Angriffe per Pufferüberlauf oder auch Konfigurationsfehler. Der Schutz ist hier stets nur als zusätzlicher Schutzmechanismus zu verstehen, der keinesfalls die Notwendigkeit ersetzt, eine    Webanwendung mit ausreichender Sicherheit zu entwickeln und auch zu testen. Im Rahmen von Tests
    sollte eine WAF dabei stets deaktiviert sein, damit diese nicht die Testergebnisse verfälscht.
    Um den größtmöglichen Nutzen der WAF zu erhalten, muss deren Konfiguration durch fachmännisches
    Personal auf die Webanwendung angepasst werden. \cite{BSI} \cite{WAF}


    \newpage

  \section{Funktionsweise WVS}
  WVS sind automatisierte Werkzeuge, die Webanwendungen - in der Regel von außerhalb - nach Sicherheitslücken wie Cross-Site Scripting, SQL-Injection, Command Injection, Path Traversal und unsicheren Serverkonfigurationen absuchen. Diese Kategorie von Werkzeugen wird häufig auch als ``Dynamic Application Security Testing (DAST) Tools'' bezeichnet \cite{OWASPtools}. Im Gegensatz zum ``Static Application Security Testing (SAST)'', bei dem der Quell-, Binär- oder Bytecode auf mögliche Implementierungs- und Konstruktionsfehler überprüft wird, testen DAST-Tools die laufende Webanwendung auf ihr Verhalten während des Betriebs. Es werden die gleichen Techniken angewendet, die auch ein realer Angreifer nutzen würde, um potentielle Sicherheitslücken zu finden.

  In der Regel werden WVS auf einem Client-Computer installiert, um dann die Webanwendung zu analysieren. Im Zuge der Analyse können folgende vier Phasen durchlaufen werden \cite{BSI4}:
  \begin{itemize}
    \item Crawl/Scan:
    Ein Web Scanner bewegt sich ähnlich wie eine Suchmaschine durch die gesamte
    Webanwendung. Dabei werden sämtliche Links besucht und gegebenenfalls Formfelder mit Testwerten ausgefüllt. Der Scanner erhält so ein umfassendes Wissen über den Aufbau der Webanwendung, die Funktionsweise von Dialogschritten und den Inhalt von Formular-Seiten.
    Die Webanwendung wird auf häufig vorhandene Installations-, Konfigurations- oder
    Testverzeichnisse, sowie bekannte problematische oder informative Dateien durchsucht (Stichwörter: Directory Enumeration, File Enumeration, Directory Traversal, Forceful Browsing).
    Unterstützt werden kann die Scan-Phase durch Einbeziehen großer öffentlicher
    Suchmaschinen. Vorausgesetzt wird hierfür, dass die eigene Website bereits durch
    diese Suchmaschinen indexiert worden ist. Die Ergebnisse dieses Schrittes wurden
    dann auf den Suchmaschinen gespeichert (Index/Cache). Im Rahmen einer Scan-Phase können diese Suchmaschinen nun gezielt nach Informationen über die eigene Website abgefragt werden. Für die Zusammenstellung von Suchmustern, sowie Automatisierung der Abfragen kann die Nutzung spezieller Tools zweckmäßig sein.
    \item Analyse:
    In einem nächsten Schritt wird die Webanwendung einer eingehenden Sicherheitsanalyse unterzogen. Die gewonnenen Erkenntnisse können in einer Datenbank hinterlegt werden: Typ und Version des Webservers, verwendete Technologien und Tools (CGI, Servlets, JSP, JavaScript, PHP, usw.), Verwendung von Cookies oder
    anderer Mechanismen zum Session-Tracking, Analyse der Form-Parameter einer Seite, insbesondere auch die Verwendung von Hidden-Parametern, Extraktion von Kommentaren.
    \item Audit/Penetrationstest:
    Unter Einbeziehung des in der vorangegangenen Phase gewonnenen Wissens werden
    systematisch fehlerhafte oder unzulässige Eingabemuster erzeugt, und an die Webanwendung versandt. Einige der existierenden Scanner unterteilen diese Phase in eine schadlose, und eine potentiell schadhafte Prüfung.
    \item Reporting:
    Die Ergebnisse der Scan- und Audit-Phase werden in Reports zusammengefasst. Der
    Umfang reicht dabei von Executive Summaries mit Nennung nur der größten Schwachstellen bis hin zu detaillierten Beschreibungen, in denen auch erste Hinweise für die Behebung des jeweiligen Problems gegeben werden können. Bei den meisten WVS hat sich ein System etabliert, bei dem die gefundenen Schwachstellen je nach Schweregrad in 4 Kategorien eingeteilt werden \cite{Lepofsky}:
    \begin{itemize}
      \item High: Schwachstellen, die es Angreifern erlauben, die komplette Kontrolle über die Webanwendung einschließlich Server zu übernehmen. Angreifer können auf die Datenbank der Anwendung zugreifen, Konten ändern und vertrauliche Informationen stehlen. XSS und SQL-Injection sind Beispiele für Schwachstellen mit hohem Schweregrad, die bei der Erkennung durch einen Scanner oberste Priorität haben sollten.

      \item Medium: Sicherheitslücken, die Angreifern den Zugriff auf ein angemeldetes Benutzerkonto ermöglichen, um vertrauliche Inhalte anzuzeigen. Angreifer erhalten Zugriff auf Informationen, mit denen sie zusätzlich andere Schwachstellen ausnutzen können, oder das System besser verstehen, damit sie ihre Angriffe verfeinern können. Open Redirection ist ein Beispiel für eine Schwachstelle mit mittlerem Schweregrad, durch die ein Angreifer einen Benutzer auf eine schädliche Website umleiten kann. Schwachstellen mit mittlerem Schweregrad sollten so schnell wie möglich behoben werden, wenn sie von einem Scanner erkannt werden.

      \item Low: Diese Schwachstellen haben nur minimalen Einfluss oder können von einem Angreifer nicht ausgenutzt werden. Cookies, die nicht als ``Http Only'' gekennzeichnet sind, sind ein Beispiel für eine Sicherheitsanfälligkeit mit niedrigem Schweregrad. Das Markieren von Cookies als Http Only macht das Cookie für clientseitige Skripts unlesbar und bietet somit eine zusätzliche Schutzschicht gegen XSS-Angriffe. Schwachstellen mit geringem Schweregrad sollten untersucht und korrigiert werden, wenn Zeit und Budget dies zulassen.

      \item Informational: Dies sind keine Schwachstellen, sondern lediglich Warnungen, die Informationen über die Webanwendung enthalten. Beispiele sind die erforderliche NTLM-Autorisierung und die Datenbankermittlung (MySQL). Für diese Informationsalarme ist keine Aktion erforderlich.
    \end{itemize}

  \end{itemize}



\chapter{Methodik}
  \section{Testaufbau}
  Für die Tests wurde der Ansatz des Black-Box Testings (siehe 2.2.2.2) verfolgt, der Ablauf des Testens war für jeden WVS identisch.
  Nach dem Scannen der 7 verwundbaren Web-Applikationen wurde jeweils ein entsprechender Bericht in Form einer HTML-Datei generiert, der die gefundenen Schwachstellen und je nach WVS auch die benötigte Zeit für den Scan auflistet. Bei den WVS, die die Scanzeit nicht im Report aufführen, wurde manuell gemessen.
  Neben diesen evidenten Daten fließen subjektive Eindrücke wie Handhabung und Bedienbarkeit der Software und die Qualität der erstellten Berichte in die Evaluation mit ein.\\
  Um dies messbar zu machen und um am Ende ein Ranking der getesteten WVS abbilden zu können, musste ein Bewertungssystem etabliert werden. Die Bewertungskategorien wurden nach Relevanz gewichtet und für die Bewertung eine Skala mit fünf Skalenwerten von 0 bis 4 Punkten herangezogen:
  \begin{table}[H]
     \begin{tabular}{ll}
     0         & \textbf{ungenügend}             \\
     1         & \textbf{unterdurchschnittlich}  \\
     2         & \textbf{durchschnittlich}       \\
     3         & \textbf{überdurchschnittlich}   \\
     4         & \textbf{überragend}
     \end{tabular}
   \end{table}
   Das Gesamtergebnis setzt sich aus den Bewertungen für die Bedienung (20\%), Qualität des Reportings (20\%), der Geschwindigkeit (10\%) und dem Scanergebnis (50\%) zusammen. Die höchste zu erreichende Punktzahl ist somit 40.\\
   Angesichts zahlreicher Abstürze der Software Acunetix unter Windows (siehe Punkt 4.2.2) wurde eine weitere Bewertungskategorie ``Stabilität'' in Betracht gezogen, aber im Hinblick auf die Ununterscheidbarkeit aller anderen stabil laufenden WVS wieder verworfen. Die Abstürze der Acunetix-Software wurden schließlich mit einem Abzug bei der Gesamtpunktzahl berücksichtigt.\\
   Für das Scanergebnis wurde die reine Anzahl der gefundenen Schwachstellen zu Grunde gelegt. Für eine tiefergehende Evaluation ist eine manuelle Validierung aller Schwachstellen auf True- und False-Positives in Erwägung zu ziehen, was jedoch angesichts der Vielzahl (über 3000) an Funden den Rahmen dieser Ausarbeitung gesprengt hätte.

  Für die Tests wurde folgendes System verwendet:
  \\Intel Core i7-8700K CPU mit 32 GB RAM
  \\Microsoft Windows 10 pro, Version 1809 (64 bit)
  \\Als Virtuelle Maschinen innerhalb von VirtualBox: Kali Linux 18.4 und Parrot OS 4.5.1, jeweils ausgestattet mit 2 Kernen und 8 GB RAM.
  \begin{figure}[H]
    \centering
    \includegraphics[width=0.9\textwidth]{Images/Visio}
    \caption[Testaufbau]{Testaufbau}
  \end{figure}
\section{Web Application Vulnerability Scanner (WVS)}
  \subsection{Auswahlkriterien}
  OWASP listet 50 verschiedene Tools zum Scannen von Webanwendungen auf
  \cite{OWASPtools}. Die Auswahl wurde auf WVS mit folgenden Eigenschaften eingegrenzt:
  \begin{itemize}
    \item 1. Free und Open Source WVS.
    \item 2. Kommerzielle WVS, die eine voll funktionsfähige Testversion anbieten.
    \item 3. WVS, deren aktuelles Release nicht älter als 2 Jahre ist oder innerhalb der letzten 2 Jahre modifiziert wurde.
    \item 4. WVS, die in der Lage sind, umfassende Scans auszuführen, um möglichst viele verschiedene Schwachstellenarten aufzuspüren.
  \end{itemize}
  Bei den bekanntesten Anbietern kommerzieller WVS wurde jeweils eine Testversion angefragt, um ein realistisches Abbild der aktuell meistgenutzten Tools zu erhalten und um die Ergebnisse der kostenlosen denen der kommerziellen Scanner gegenüber zu stellen. Es handelt sich um folgende Firmen:
  \\Acunetix, Beyond Security (WSSA), Beyond Trust (Retina), Netsparker, N-Stalker, Portswigger (BurpSuite Pro), Rapid 7 (Nexpose) und Tenable (Nessus).

  Von den angefragten Firmen stellten Acunetix, Netsparker, N-Stalker, Portswigger und Tenable jeweils eine Testversion zur Verfügung.

  \subsection{Ausgewählte WVS}
  Aus der Vielzahl der ursprünglich in Betracht kommenden WVS haben sich am Ende 9 herauskristallisiert, 5 mit einer OpenSource-Lizenz und 4 kommerzielle Produkte. Unter Punkt 4.2 werden sie im Zuge der Evaluation näher beschrieben.
    \subsubsection{Free und Open Source WVS}
      \begin{table}[H]
        \centering
          \begin{tabular}{|l|l|c|c|}
            \hline
            \textbf{WVS}              & \textbf{Entwickler}  & \textbf{Version}     & \textbf{Verwendete Plattform}  \\
            \hline
            \textbf{Arachni}          & Tasos Laskos         & 0.5.12 (WebUI)       & Windows                       \\
            \hline
            \textbf{Nikto}            & cirt.net             & 2.1.6                & Kali                          \\
            \hline
            \textbf{OpenVAS}          & Greenbone            & 7.0.3                & Parrot                        \\
            \hline
            \textbf{Wapiti}           & devloop              & 3.0.1                & Kali                          \\
            \hline
            \textbf{Zed Attack Proxy} & OWASP                & 2.7.0                & Parrot                        \\
            \hline
          \end{tabular}
        \caption[Ausgewählte Free und Open Source WVS]{Ausgewählte Free und Open Source WVS}
      \end{table}

    \subsubsection{Kommerzielle WVS}
      \begin{table}[H]
        \centering
          \begin{tabular}{|p{4cm}|l|c|c|}
            \hline
            \textbf{WVS}            & \textbf{Anbieter} & \textbf{Version} & \textbf{Verwendete Plattform}  \\
            \hline
            \textbf{Acunetix}       & Acunetix          & 12.0.190206130   & Windows/Kali                          \\
            \hline
            \textbf{Burp Suite Pro} & Portswigger       & 2.0.15           & Windows                       \\
            \hline
            \textbf{Nessus}         & Tenable           & 8.2.2            & Windows                       \\
            \hline
            \textbf{Netsparker}     & Netsparker        & 5.2.0.22027      & Windows                       \\
            \hline
          \end{tabular}
        \caption[Ausgewählte Kommerzielle WVS]{Ausgewählte Kommerzielle WVS}
      \end{table}
\newpage
  \subsection{Nicht ausgewählte WVS}
    Nachfoldend werden alle WVS aufgelistet, die als Kandidaten in Erwägung gezogen wurden, bei näherer Betrachtung jedoch nicht die erforderlichen Voraussetzungen erfüllten, um in die Evaluation aufgenommen zu werden.
    \subsubsection{Free und Open Source WVS}
    \begin{itemize}
      \item GoLismero \cite{GoLismero}: GoLismero ist ein in Python geschriebenes Framework, das verschiedene Penetrationtesting-Tools in sich vereint. Theoretisch sollten bei einem Angriff alle Tools angewendet und die jeweiligen Ergebnisse in einem einzigen Report gebündelt werden. In der Praxis hängte sich das Programm jedoch jedes Mal nach einer Weile auf, sowohl unter Kali-Linux und Parrot OS, als auch unter Windows. Es werden zwar Teilergebnisse auf dem Bildschirm angezeigt, dies reicht aber nicht aus, um in die Evaluation aufgenommen zu werden.
      \item Grabber \cite{Grabber}: Das von Romain Gaucher entwickelte Programm ist zwar noch Bestandteil von Kali-Linux, ist aber schon über 12 Jahre alt (Latest Release 2006).
      \item Grendel-Scan \cite{Grendel}: Seit 2013 gibt es auf dem SourceForge-Repository von David Byrne keine Veränderung.
      \item Iron Wasp \cite{Iron}: Die aktuelle Verison des Windows-Programms von Lavakumar Kuppan ist ein Beta-Release aus dem Jahr 2015.
      \item Ratproxy \cite{Ratproxy}: Die Entwicklung wurde 2009 von Google eingestellt.
      \item Skipfish \cite{Skipfish}: Ein weiteres Projekt von Google, das 2012 eingestellt wurde.
      \item SQLmap \cite{SQLmap}: SQLmap ist ein beliebtes Tool zum Auffinden von SQLi Schwachstellen, ist aber darauf beschränkt.
      \item Vega \cite{Vega}: Das aktuelle Release ist aus dem Jahr 2014, seit diesem Jahr wird in regelmäßigen Abständen angekündigt, ein Feature zum Exportieren der Ergebnisse hinzuzufügen, aber die Firma Subgraph scheint das Projekt nicht weiter zu verfolgen.
      \item Watobo \cite{Watobo}: Die letzte Änderung des OpenSource Scanners stammt aus dem Jahr 2015.
      \item Webscarab \cite{Webscarab}: Der Vorgänger von OWASPs Zed Attack Proxy ist veraltet (Latest Release 2011), OWASP empfiehlt, auf ZAP umzusteigen.
      \item Wfuzz \cite{Wfuzz}: Der in Python geschriebene Scanner verwendet die Bibliothek Pycurl für HTTP-Requests, diese unterstützt keine SSL/TLS Verschlüsselung.
      \item W3af \cite{W3af}: W3af wird zwar noch sporadisch mit Bibliotheks-Aktualisierungen gepflegt, das aktuelle Release ist jedoch aus dem Jahr 2014 und es ist weder auf Kali-Linux noch auf Parrot OS gelungen, alle für den Programmstart benötigten Dependencies zu installieren. Das bei der Installation generierte Script versucht, teils veraltete Module zu installieren, manuelles Nachinstallieren der aktuellen Versionen brachte keinen Erfolg.
      \begin{figure}[H]
        \centering
        \includegraphics[width=0.9\textwidth]{Images/w3af}
        \caption[W3af: Fehlende Module]{W3af: Fehlende Module}
      \end{figure}
      \item Wikto \cite{Wikto}: Wikto ist eine Windows-Portierung von Nikto und bedarf daher keiner eigenen Evaluation.
      \item Xenotix \cite{Xenotix}: Xenotix wurde von OWASP für das Auffinden von Cross-Site-Scripting Schwachstellen entwickelt und ist darauf beschränkt.
    \end{itemize}
    \subsubsection{Kommerzielle WVS}
      \begin{itemize}
        \item N-Stalker \cite{Stalker}: Die angebotene ``7-Day Evaluation Licence'' erlaubt nur das Scannen einer einzigen, vorher festgelegten URL und ist daher für den geplanten Testaufbau nicht geeignet.
      \end{itemize}
\newpage
\section{Verwundbare Web-Applikationen}
  Bei der Auswahl der Web-Applikationen musste darauf geachtet werden, dass sie für WVS geeignet sind. Auf ursprünglich in die Auswahl aufgenommene Applikationen wie WebGoat, JuiceShop (beide von OWASP) oder Damn Vulnerable Web Application (Bestandteil von Metasploitable) wurde am Ende verzichtet, da hier beim Scannen keine hinreichenden Ergebnisse hervorgebracht wurden. Diese Anwendungen  sind zwar sehr gut dokumentiert, aber hauptsächlich für das Erlernen von manuellen Angriffen entwickelt worden. Es wurde bei der Auswahl auf das OWASP Vulnerable Web Applications Directory Project zurückgegriffen, das eine Reihe von verwundbaren Web-Applikationen auflistet \cite{OWASPWebApps}. Die Auswahl deckt mehrere Technologien wie PHP, ASP.Net oder HTML5 ab.

  Nachfolgend werden die für die Auswertung genutzten Web-Applikationen kurz vorgestellt.
  \begin{itemize}
    \item WA1: Altoro Mutual\\
    Altoro Mutual ist eine in C\texttt{\#} .NET geschriebene Online-Banking Web-Applikation, die von IBM entwickelt wurde, um WVS zu testen \cite{Altoro}.
    \item WA2: Webscantest\\
    Diese Web-Applikation ist in PHP geschrieben und wurde von NTOSpider entwickelt, um WVS zu testen. Die Schwachstellen sind direkt auf der Seite dokumentiert \cite{Webscantest}.
    \item WA3: Zero Bank\\
    Eine weitere Online-Banking Web-Applikation, entwickelt von Hewlett-Packard/Micro Focus \cite{Zero}.
    \item WA4: Bitcoin Web Site\\
    Diese von Netsparker entwickelte Applikation ist in ASP.NET geschrieben und simuliert eine Online-Seite für Bitcoin-Transaktionen \cite{Aspnet}.
    \item WA5: Acuart\\
    Eine Test-Seite von Acunetix, der einen Online-Shop für Kunstwerke simuliert, geschrieben in PHP \cite{Acuart}.
    \item WA6: Crack Me Bank\\
    Eine in PHP geschriebene Online-Banking Seite, entwickelt von Trustwave \cite{CrackMeBank}:
    \item WA7: Security Tweets\\
    Security Tweets ist eine von Twitter inspirierte Social Networks Applikation, die von Acunetix entwickelt wurde und HTML5 verwendet \cite{Tweets}.
 \end{itemize}
\chapter{Evaluation}
  \section{Gefundene Schwachstellen per Webanwendung}
   H, M, L und I: High, Medium, Low und Informational. Aus Gründen der Vergleichbarkeit wurden die Werte aus der zusätzlichen Kategorie ``Critical'' bei Nessus und Netsparker mit in die Kategorie High übernommen. WA1-7: Verwundbare Web-Applikationen.
    \begin{table}[H]
      \begin{tabular}{|r|c|c|c|c|c|c|c|c|c|c|c|c|c|c|}
        \cline{2-15}
        \multicolumn{1}{r|}{}       & \multicolumn{4}{c|}{\textbf{Arachni}} & \textbf{Nikto} & \multicolumn{4}{c|}{\textbf{OpenVAS}} & \textbf{Wapiti} & \multicolumn{4}{c|}{\textbf{ZAP}}  \\
        \cline{2-15}
        \multicolumn{1}{r|}{}       & H & M & L & I              &                & H & M & L & I              &                 & H & M & L & I                        \\
        \hline
        \textbf{WA1}      & 9    & 4    & 2   & 5                 & 12             & 0    & 2    & 0   & 24                & 10              & 1    & 2    & 5   & 0                           \\
        \hline
        \textbf{WA2}        & 4    & 10   & 5   & 29                & 18             & 0    & 8    & 0   & 41                & 12              & 1    & 4    & 11  & 0                           \\
        \hline
        \textbf{WA3}          & 4    & 6    & 4   & 4                 & 18             & 6    & 51   & 2   & 34                & 2               & 0    & 1    & 2   & 0                           \\
        \hline
        \textbf{WA4}    & 22   & 6    & 8   & 27                & 15             & 0    & 4    & 0   & 24                & 32              & 3    & 4    & 6   & 0                           \\
        \hline
        \textbf{WA5}             & 56   & 6    & 10  & 24                & 18             & 2    & 31   & 1   & 77                & 31              & 4    & 1    & 2   & 0                           \\
        \hline
        \textbf{WA6}      & 29   & 10   & 4   & 10                & 9              & 0    & 11   & 1   & 29                & 6               & 4    & 3    & 3   & 0                           \\
        \hline
        \textbf{WA7}    & 10   & 2    & 3   & 6                 & 7              & 1    & 31   & 1   & 77                & 0               & 0    & 1    & 5   & 0                           \\
        \hline
        \textbf{Subt.}           & 134  & 44   & 36  & 105               & 97             & 9    & 138  & 5   & 306               & 93              & 13   & 16   & 34  & 0                           \\
        \hline
        \textbf{Total}              & \multicolumn{4}{c|}{\textbf{319}}     & \textbf{97}    & \multicolumn{4}{c|}{\textbf{458}}     & \textbf{93}     & \multicolumn{4}{c|}{\textbf{63}}                \\
        \hline
      \end{tabular}
      \caption[Gefundene Schwachstellen der Open Source WVS]{Gefundene Schwachstellen der Open Source WVS}
    \end{table}

    \begin{table}[H]
      \begin{tabular}{|r|c|c|c|c|c|c|c|c|c|c|c|c|c|c|c|c|}
        \cline{2-17}
        \multicolumn{1}{l|}{}    & \multicolumn{4}{c|}{\textbf{Acunetix}}                   & \multicolumn{4}{c|}{\textbf{BurpSuite Pro}}          & \multicolumn{4}{c|}{\textbf{Nessus}}                  & \multicolumn{4}{c|}{\textbf{Netsparker}}                 \\
        \cline{2-17}
        \multicolumn{1}{c|}{}    & H            & M            & L           & I            & H           & M          & L           & I            & H           & M           & L          & I            & H            & M           & L           & I             \\
        \hline
        \textbf{WA1}   & 4            & 10           & 2           & 34           & 10          & 0          & 7           & 18           & 0           & 3           & 2          & 21           & 4            & 6           & 10          & 15            \\
        \hline
        \textbf{WA2}     & 25           & 41           & 6           & 14           & 5           & 2          & 4           & 160          & 1           & 7           & 1          & 25           & 37           & 18          & 24          & 20            \\
        \hline
        \textbf{WA3}       & 19           & 23           & 20          & 24           & 0           & 0          & 1           & 13           & 14          & 31          & 1          & 23           & 7            & 6           & 16          & 14            \\
        \hline
        \textbf{WA4} & 73           & 26           & 6           & 7            & 17          & 1          & 4           & 119          & 1           & 6           & 1          & 20           & 21           & 8           & 16          & 28            \\
        \hline
        \textbf{WA5}          & 39           & 34           & 9           & 19           & 28          & 0          & 4           & 49           & 21          & 18          & 1          & 21           & 24           & 18          & 12          & 12            \\
        \hline
        \textbf{WA6}   & 6            & 14           & 27          & 5            & 11          & 0          & 1           & 74           & 0           & 5           & 1          & 22           & 14           & 4           & 11          & 13            \\
        \hline
        \textbf{WA7} & 14           & 4            & 9           & 3            & 8           & 2          & 2           & 11           & 0           & 1           & 2          & 16           & 8            & 2           & 10          & 11            \\
        \hline
        \textbf{Subt.}        & \textbf{180} & \textbf{152} & \textbf{79} & \textbf{106} & \textbf{79} & \textbf{5} & \textbf{23} & \textbf{444} & \textbf{37} & \textbf{71} & \textbf{9} & \textbf{148} & \textbf{115} & \textbf{62} & \textbf{99} & \textbf{113}  \\
        \hline
        \textbf{Total}           & \multicolumn{4}{c|}{\textbf{517}}                        & \multicolumn{4}{c|}{\textbf{551}}                     & \multicolumn{4}{c|}{\textbf{265}}                     & \multicolumn{4}{c|}{\textbf{389}}                        \\
        \hline
      \end{tabular}
      \caption[Gefundene Schwachstellen der kommerziellen WVS]{Gefundene Schwachstellen der kommerziellen WVS}
    \end{table}

  \section{Bedienung, Reporting und Geschwindigkeit}
    \subsection{Open Source}
        \begin{itemize}
          \item Arachni \cite{Arachni}:\\
            Arachni gibt es als reine Terminal-Anwendung oder als Ruby on Rails Framework mit Web-Interface. Die Web-Oberfläche ist übersichtlich und verständlich aufgebaut, der User findet sich schnell zurecht und kann sofort mit dem Scannen einer Seite beginnen, die Scangeschwindigkeit ist überdurchschnittlich. Als Hilfestellung gibt es ein umfangreiches Wiki mit Erklärungen und Screenshots. Der Report ist sehr umfangreich, verschiedene Balken- und Kuchendiagramme geben Statistiken über Art und Schweregrad der Funde wieder, zudem gibt es Verlinkungen zu den OWASP Top 10 und detaillierte Ausführungen über die Schwachstellen. Arachni unterscheidet bei den Funden außerdem zwischen gesicherten (``Trusted'') und noch zu überprüfenden Ergebnissen (``Untrusted'').

            \textbf{Bewertung: Reporting 4, Bedienung 3, Geschwindigkeit 3}
            \begin{figure}[H]
              \centering
              \includegraphics[width=0.9\textwidth]{Images/Arachni}
              \caption[Arachni WebUI 0.5.12]{Arachni WebUI 0.5.12}
            \end{figure}
          \item Nikto \cite{Nikto}:\\
            Nikto ist ein gut dokumentiertes Terminal-Programm, nach kurzer Einarbeitung hat ein ungeübter User die benötigten Befehle und Optionen gefunden, um einen Scan zu starten.
            Der generierte Report listet alle gefundenen Schwachstellen auf, unterscheidet diese jedoch im Gegensatz zu den meisten anderen WVS nicht zwischen High, Medium, Low und Informational. Die Scan-Geschwindigkeit ist überdurchschnittlich.

            \textbf{Bewertung: Reporting 1, Bedienung 2, Geschwindigkeit 3}
          \item OpenVAS \cite{OpenVAS}:\\
            OpenVAS ist aus der Software Nessus hervorgegangen, als diese im Jahr 2005 von Open Soucre zu einer kommerziellen Lizenz wechselte, und wird seitdem auf Basis der letzten freien Nessus-Version 2.2 von Greenbone Networks weiterentwickelt. Der Scanner ist eingebettet in den Greenbone Security Assistant, der über ein Web-Interface bedient wird. Die Oberfläche ist nicht selbsterklärend, es bedarf etwas an Recherche, bis sich dem User der logische Aufbau des Programms erschließt. Hier ist das ``Tech Doc-Portal'' von Greenbone sehr hilfreich. Zuerst muss unter Configuration/Targets ein Ziel definiert werden, dann kann der User für dieses Ziel unter dem Punkt ``Scans'' einen Task erstellen und diesen entweder sofort oder per Schedule starten.
            Die umständliche Handhabung hat jedoch den Vorteil, dass sich leicht mehrere Tasks automatisieren lassen. Der Bericht präsentiert sich etwas sparsam, die gefundenen Schwachstellen der Kategorie ``Informational'' werden nicht aufgelistet.

            \textbf{Bewertung: Reporting 2, Bedienung 1, Geschwindigkeit 2}
            \begin{figure} [H]
              \centering
              \includegraphics[width=0.85\textwidth]{Images/OpenVAS}
              \caption[OpenVAS im Greenbone Security Assistant]{OpenVAS im Greenbone Security Assistant}
            \end{figure}
          \item Wapiti \cite{Wapiti}:\\
            In der Handhabung und im Reporting ähnelt Wapiti dem anderen reinen Terminal-Programm Nikto. Die auswählbaren Optionen sind ähnlich, und die gefundenen Schwachstellen werden auch hier nicht in Kategorien eingeteilt. Die Dokumentation fällt etwas spartanischer aus, ist aber ausreichend, um sich schnell zurecht zu finden. Wapiti scannt fast so schnell wie Nikto und reiht sich hier auf dem dritten Platz ein.

            \textbf{Bewertung: Reporting 1, Bedienung 2, Geschwindigkeit 3}

          \item Zed Attack Proxy \cite{ZAP}:\\
            Der von OWASP entwickelte Scanner hat eine übersichtliche Benutzeroberfläche und lässt sich intuitiv bedienen. Auf der Startseite lässt sich direkt die anzugreifende URL eingeben und ohne weitere Konfiguration angreifen. Es gibt umfangreiche Hilfestellung in Form eines Handbuchs und einem Online-Wiki, zudem gibt es mit der OWASP ZAP User Group ein gut frequentiertes Benutzerforum, auf dem ein reger Austausch zwischen den Benutzern stattfindet.
            Auffällig sind die langen Scan-Zeiten von mehreren Stunden, die sich jedoch nicht in einer höheren Anzahl an gefundenen Schwachstellen widerspiegeln. Die wenigen Funde werden im Report ausführlich beschrieben einschließlich umfangreicher Empfehlungen zur Behebung.

            \textbf{Bewertung: Reporting 3, Bedienung 3, Geschwindigkeit 0}
          \begin{figure}[H]
            \centering
            \includegraphics[width=0.9\textwidth]{Images/ZAP}
            \caption[Grafische Benutzeroberfläche von ZAP]{Grafische Benutzeroberfläche von ZAP}
          \end{figure}
        \end{itemize}

    \subsection{Kommerziell}
      \begin{itemize}
        \item Acunetix \cite{Acunetix}:\\
          Acunetix stellte eine 14-tägige Vollversion zur Verfügung. Die graphische Oberfläche ist sehr übersichtlich und lässt sich intuitiv bedienen. Online gibt es ein Support-Portal mit ausführlicher Dokumentation und Hilfestellung.
          Auffällig ist die im Vergleich zu allen anderen WVS ungewöhnlich hohe Scan-Geschwindigkeit, die sich im Minutenbereich einordnet.
          Unter Windows brachte Acunetix das System mehrmals zum Absturz (BSoD), so dass auf die Linux-Version ausgewichen wurde. Hier lief das Programm stabil und scannte von den kommerziellen WVS am schnellsten, von allen WVS muss sich Acunetix hier nur den Terminalprogrammen Wapiti und Nikto geschlagen geben. Der Report listet zu jeder gefundenen Schwachstelle sehr detailliert die vollständigen GET- und POST-Requests sowie teilweise seitenlange Code-Passagen auf. Die Empfehlungen zur Behebung der Funde könnten hingegen ausführlicher sein. Aufgrund der Abstürze wurden bei der Gesamtpunktzahl 3 Punkte abgezogen.

          \textbf{Bewertung: Reporting 3, Bedienung 3, Geschwindigkeit 4}\\

          \begin{figure}[H]
            \centering
            \includegraphics[width=0.9\textwidth]{Images/Acunetix}
            \caption[Weboberfläche von Acunetix]{Weboberfläche von Acunetix}
          \end{figure}
          \newpage
        \item BurpSuite Pro \cite{Burp}:\\
          Die Firma Portswigger stellte auf Anfrage eine 30-tägige unbeschränkte Testversion zur Verfügung.
          BurpSuite Pro enthält eine umfangreiche Dokumentation mit zahlreichen Hilfestellungen für verschiedene Anwendungsszenarien.  Das Dashboard ist sehr übersichtlich und intuitiv zu bedienen: mit Hilfe des Buttons ``New Scan'' lässt sich direkt ohne größeren Konfigurationsaufwand eine Website erfolgreich und in durchschnittlicher Geschwindigkeit scannen.
          \begin{figure}[H]
            \centering
            \includegraphics[width=0.91\textwidth]{Images/BurpSuitePro}
            \caption[Dashboard von BurpSuite Pro]{Dashboard von BurpSuite Pro}
          \end{figure}
          Der von BurpSuite generierte Report ist von allen getesteten WVS der umfangreichste und ist trotzdem sehr übersichtlich. Ähnlich wie Arachni unterscheidet BurpSuite zwischen gesicherten und ungesicherten Ergebnissen, allerdings noch genauer: in einer ``Confidence''-Tabelle wird zwischen gesicherten (Certain), wahrscheinlichen (Firm) und möglichen (Tentative) Schwachstellen unterschieden.

          \textbf{Bewertung: Reporting 4, Bedienung 3, Geschwindigkeit 2}
          \begin{figure}[H]
            \centering
            \includegraphics[width=0.76\textwidth]{Images/BurpReport}
            \caption[Confidence-Tabelle im Report von BurpSuite Pro]{Confidence-Tabelle im Report von BurpSuite Pro}
          \end{figure}

        \item Nessus \cite{Nessus}:\\
          Die Firma Tenable bietet zwei Testversionen an: Nessus Home für das Testen des heimischen Netzwerks, gültig für ein Jahr sowie eine unbeschränkte Version von Nessus Pro, gültig für sieben Tage.
          Nessus präsentiert sich auf einer übersichtlich gestalteten Web-Schnittstelle und ist selbsterklärend bedienbar. Einen Hilfe-Button sucht man vergeblich, die umfangreiche Dokumentation mit Anleitungen findet man online durch Recherchieren. Die Scangeschwindigkeit ist durchschnittlich, der Report übersichtlich aber eher spartanisch. Nur durch Anklicken der Funde gibt es online Beschreibungen und Lösungsvorschläge.

          \textbf{Bewertung: Reporting 2, Bedienung 3, Geschwindigkeit 2}
          \begin{figure}[H]
            \centering
            \includegraphics[width=0.9\textwidth]{Images/Nessus}
            \caption[Grafische Benutzeroberfläche von Nessus]{Grafische Benutzeroberfläche von Nessus}
          \end{figure}
        \item Netsparker \cite{Netsparker}:\\
          Netsparker stellte nach Rücksprache eine 14-tägige Testversion zur Verfügung, die auf 8 zu scannende Web-Applikationen begrenzt ist. Die grafische Benutzeroberfläche ist etwas unruhig und überladen, der User kann aber direkt über den Button ``New'' eine Web-Seite scannen. Die Scangeschwindigkeit ist im unteren Mittelfeld angesiedelt. Der Report ist äußerst umfangreich aber etwas unübersichtlich. Wie Nessus hat auch Netsparker nach High, Medium, Low und Informational eine fünfte Kategorie ``Critical'' eingeführt, die auf besonders kritische Schwachstellen hinweist. Wie bei Arachni und BurpSuite werden gesicherte Ergebnisse gesondert gekennzeichnet, hier als ``Confirmed''.

          \textbf{Bewertung: Reporting 3, Bedienung 2, Geschwindigkeit 1}
          \begin{figure}[H]
            \centering
            \includegraphics[width=0.9\textwidth]{Images/Netsparker}
            \caption[Grafische Benutzeroberfläche von Netsparker]{Grafische Benutzeroberfläche von Netsparker}
          \end{figure}
     \end{itemize}
\newpage
     \subsection{Scanzeiten}
     In den beiden folgenden Tabellen sind die Zeiten aufgeführt, die die WVS benötigten, umd die jeweiligen Webanwendungen zu scannen.
     \paragraph{Open Source WVS}
     \
\begin{table}[H]
\begin{tabular}{|r|c|c|c|c|c|}
\hline
                            & \textbf{Arachni} & \textbf{Nikto} & \textbf{OpenVAS} & \textbf{Wapiti} & \textbf{Zed Attack Proxy}  \\
\hline
\textbf{Altoro Mutual}      & 41               & 96             & 38               & 88              & 615                        \\
\hline
\textbf{Webscantest}        & 86               & 80             & 90               & 82              & 821                        \\
\hline
\textbf{Zero Bank}          & 42               & 29             & 85               & 31              & 786                        \\
\hline
\textbf{Aspnet Testsparker} & 28               & 20             & 29               & 14              & 579                        \\
\hline
\textbf{Acuart}             & 39               & 9              & 43               & 23              & 660                        \\
\hline
\textbf{Crack Me Bank}      & 38               & 32             & 45               & 28              & 704                        \\
\hline
\textbf{Security Tweets}    & 28               & 19             & 33               & 27              & 523                        \\
\hline
\textbf{Total}              & \textbf{302}     & \textbf{285}   & \textbf{363}     & \textbf{293}    & \textbf{4688}              \\
\hline
\end{tabular}
\caption[Scanzeiten der Open Source WVS in Minuten]{Scanzeiten der Open Source WVS in Minuten}
\end{table}
\paragraph{Kommerzielle WVS}
\

\begin{table}[H]
\begin{tabular}{|r|c|c|c|c|}
\hline
                            & \textbf{Acunetix} & \textbf{Burp Suite Pro} & \textbf{Nessus} & \textbf{Netsparker}  \\
\hline
\textbf{Altoro Mutual}      & 15                & 56                      & 44              & 46                   \\
\hline
\textbf{Webscantest}        & 60                & 102                     & 82              & 241                  \\
\hline
\textbf{Zero Bank}          & 26                & 71                      & 68              & 47                   \\
\hline
\textbf{Aspnet Testsparker} & 11                & 30                      & 27              & 52                   \\
\hline
\textbf{Acuart}             & 6                 & 45                      & 61              & 37                   \\
\hline
\textbf{Crack Me Bank}      & 28                & 39                      & 31              & 71                   \\
\hline
\textbf{Security Tweets}    & 8                 & 41                      & 51              & 20                   \\
\hline
\textbf{Total}              & \textbf{154}      & \textbf{384}            & \textbf{364}    & \textbf{514}         \\
\hline
\end{tabular}
\caption[Scanzeiten der kommerziellen WVS in Minuten]{Scanzeiten der kommerziellen WVS in Minuten}
\end{table}


     \newpage
     \subsection{Gesamtbewertung und Ranking}
     \paragraph{Open Source WVS}
     \
       \begin{table}[H]
       \begin{tabular}{|r|c|c|c|c|c|}
       \hline
       \textbf{}            & \textbf{Arachni} & \textbf{Nikto} & \textbf{OpenVAS} & \textbf{Wapiti} & \textbf{ZAP}  \\
       \hline
       \textbf{Scanergebnis (50\%) *5}    & \textbf{3}      &  \textbf{1}      & \textbf{4}      &   \textbf{1}     &  \textbf{0}       \\
       \hline
       \textbf{Reporting (20\%) *2}       &  \textbf{4}     &  \textbf{1}      & \textbf{2}      &   \textbf{1}    &   \textbf{3}     \\
       \hline
       \textbf{Bedienung (20\%) *2}       & \textbf{3}       & \textbf{2}     & \textbf{1}       & \textbf{2}      & \textbf{3}        \\
       \hline
       \textbf{Geschwindigkeit (10\%)} & \textbf{3}       & \textbf{3}    & \textbf{2}       & \textbf{3}     & \textbf{0}          \\
       \hline
       \textbf{Gesamt}                 &   \textbf{32}     &  \textbf{15}   &  \textbf{28}      &  \textbf{15}    &  \textbf{10}          \\
       \hline
       \end{tabular}
       \caption[Bewertung der Open Source WVS]{Bewertung der Open Source WVS}
     \end{table}
     \paragraph{Kommerzielle WVS}
     \
       \begin{table}[H]
       \begin{tabular}{|r|c|c|c|c|}
       \hline
       \textbf{}       & \textbf{Acunetix} & \textbf{BurpSuite Pro} & \textbf{Nessus} & \textbf{Netsparker}  \\
       \hline
       \textbf{Scanergebnis (50\%) *5}    & \textbf{4}      &  \textbf{4}            & \textbf{2}      &   \textbf{3}         \\
       \hline
       \textbf{Reporting (20\%) *2}       &  \textbf{3}     &  \textbf{4}            & \textbf{2}      &   \textbf{3}         \\
       \hline
       \textbf{Bedienung (20\%) *2}       & \textbf{3}      & \textbf{3}           & \textbf{3}      & \textbf{2}        \\
       \hline
       \textbf{Geschwindigkeit (10\%)} & \textbf{4}       & \textbf{2}              & \textbf{2}      & \textbf{1}            \\
       \hline
       \textbf{Gesamt}                 &  \textbf{36}     & \textbf{36}              & \textbf{22}       &  \textbf{26}                    \\
       \hline
       \end{tabular}
       \caption[Bewertung der kommerziellen WVS]{Bewertung der kommerziellen WVS}
     \end{table}
     \paragraph{Ranking aller WVS}
     \
     \begin{table}[H]
  \begin{tabular}{|r|r|c|c|c|c|c|}
 \cline{2-7}
 \multicolumn{1}{l|}{} & \textbf{WVS}           & \textbf{Gesamt} & Scanergebnis & Reporting & Bedienung & Geschwindigkeit  \\
 \hline
 \textbf{1}            & \textbf{BurpSuite Pro} & \textbf{36}     & 4            & 4         & 3         & 2                \\
 \hline
 \textbf{2}            & \textbf{Acunetix}      & \textbf{33}\footnotemark     & 4            & 3         & 3         & 4                \\
 \hline
 \textbf{3}            & \textbf{Arachni}       & \textbf{32}     & 3            & 4         & 3         & 3                \\
 \hline
 \textbf{4}            & \textbf{OpenVAS}       & \textbf{28}     & 4            & 2         & 1         & 2                \\
 \hline
 \textbf{5}            & \textbf{Netsparker}    & \textbf{26}     & 3            & 3         & 2         & 1                \\
 \hline
 \textbf{6}            & \textbf{Nessus}        & \textbf{22}     & 2            & 2         & 3         & 2                \\
 \hline
 \textbf{7}            & \textbf{Nikto}         & \textbf{14}     & 1            & 1         & 2         & 3                \\
 \hline
 \textbf{7}            & \textbf{Wapiti}        & \textbf{14}     & 1            & 1         & 2         & 3                \\
 \hline
 \textbf{9}            & \textbf{ZAP}           & \textbf{12}     & 0            & 3         & 3         & 0                \\
 \hline
 \end{tabular}
 \caption[Ranking aller WVS]{Ranking aller WVS}
 \end{table}
 \footnotetext[1]{Ergebnis nach Abzug von 3 Punkten}

\chapter{Diskussion}
  \section{Anzahl der gefundenen Schwachstellen}
  Die Tabellen mit den gefundenen Schwachstellen zeigen eine große Bandbreite an unterschiedlichen Ergebnisssen. Während Zed Attack Proxy insgesamt nur 63 Schwachstellen findet, sind es bei BurpSuite Pro mit 551 Funden ungefähr 9 mal so viele. Die beiden Terminalprogramme Nikto und Wapiti liegen fast gleich auf mit 97 und 93 Funden und sind damit nur geringfügig besser als das Schlusslicht Zed Attack Proxy.
  Mit 265 Funden hat Nessus bei den kommerziellen Scannern das schlechteste Ergebnis, überraschend ist hier vor allem, dass die Open Source Software und Nessus-Abspaltung OpenVAS fast 200 Schwachstellen mehr findet.
  OpenVAS lässt mit diesem Ergebnis auch noch den kommerziellen WVS Netsparker (389 Funde) hinter sich und belegt insgesamt den dritten Platz bei der Anzahl gefundener Schwachstellen. Arachni findet 319 Schwachstellen und reiht sich damit zwischen Nessus und Netsparker ein. Acunetix muss sich mit 517 Funden nur BurpSuite Pro geschlagen geben und belegt hier den zweiten Platz.

  Auch wenn die Werte in die Katgorien High, Medium, Low und Informational aufgesplittet werden, gibt es eine breite Palette an Ergebnissen, es gibt fast keinerlei Überein- stimmungen oder erkennbare Muster. Bei den WVS, die bei den insgesamt gefundenen Schwachstellen ungefähr gleich auf liegen, lässt sich dies veranschaulichen: Acunetix hat zum Beispiel mehr als 330 Funde in High und Medium eingeteilt und ca. 180 in Low und Informational, BurpSuite Pro stuft die Schwachstellen als nicht so schwerwiegend ein,  gerade einmal 84 Funde finden sich in High und Medium, aber 467 in Low und Informational. Arachni hat 134 Funde in die Kategorie High eingestuft, Nessus nur 37 und OpenVAS gar nur 9.

  Selbst wenn man die Ergebnisse nach den verwundbaren Webanwendungen aufteilt, gibt es keine Wiedererkennungswerte, weder gibt es eine Webseite, bei der von allen WVS besonders viele Schwachstellen gefunden wurden, noch sticht eine mit besonders wenigen Funden heraus. Auch die Anzahl der Funde pro Webanwendung ist von WVS zu WVS sehr verschieden. So findet zum Beispiel Acunetix auf der Seite Zero Bank (WA3) 68 Schwachstellen, BurpSuite Pro nur 14, andererseits findet BurpSuite Pro auf Webscantest (WA2) 169 Schwachstellen und Acunetix nur 86. OpenVAS findet auf Security Tweets (WA7) 110 Schwachstellen, Netsparker nur 31. Auf Webscantest (WA2) hingegen findet Netsparker doppelt so viele (99) Schwachstellen wie OpenVAS (49).\\
  Bezüglich der Webanwendungen gibt es eine interessante Beobachtung:  Acuart (WA5) und Security Tweets (WA7) wurden beide von Acunetix entwickelt, die naheliegende Vermutung, dass der WVS von Acunetix hier besonders viele Schwachstellen finden müsste, hat sich nicht bestätigt.
  Ebensowenig findet auch Netsparker auf der selbst entwickelten Bitcoin Web Site (WA4) relevant mehr Schwachstellen als die anderen WVS, Acunetix und BurpSuite Pro finden sogar deutlich mehr.

  Zusammenfassend muss festgehalten werden, dass bei der Anzahl der gefundenen Schwachstellen keinerlei Regelmäßigkeiten erkennbar sind, und so lässt sich die erste Forschungsfrage \textbf{Wie viele Schwachstellen werden von den WVS gefunden?} nur mit einem Blick auf die Tabelle beantworten - in absoluten Zahlen pro WVS.

  \section{Bedienung, Reporting und Scan-Geschwindigkeit}
  Dieses Kapitel widmet sich der Fragestellung \textbf{Wie unterscheiden sich die WVS in den Kategorien Bedienung, Reporting und Scan-Geschwindigkeit?}
  \subsection{Bedienung}
  In dieser Kategorie gibt es keinen WVS, der die volle Punktzahl erreicht, es gibt aber auch keinen mit 0 Punkten. Den beiden Terminalprogrammen Nikto und Wapiti fehlt es naturgemäß am Komfort, den eine grafische Benutzeroberfläche bietet, trotzdem schneiden sie noch besser ab als OpenVAS, das erst nach umfangreicher Einarbeitungszeit sinnvoll genutzt werden kann. Das Streben nach Automatisierung geht hier zu Lasten der Bedienbarkeit. Netsparker befindet sich mit seiner überladenen Benutzeroberfläche ebenfalls nur im Mittelfeld, Acunetix, BurpSuite Pro und Nessus sind - erwartungsgemäß für kommerzielle WVS - sehr übersichtlich und benutzerfreundlich.
  Die Open Source Programme Arachni und Zed Attack Proxy stehen dem jedoch in nichts nach und sind ebenfalls leicht verständlich und intuitiv zu bedienen. \\
  Insgesamt gibt es in dieser Kategorie keinen klaren Sieger, mit Ausnahme von OpenVAS sind alle WVS mühelos und unproblematisch zu bedienen.
  \subsection{Reporting}
  Nikto und Wapiti schneiden in dieser Kategorie unterdurchschnittlich ab, da sie die gefundenen Schwachstellen nicht nach Schweregrad unterteilen. Das Reporting von OpenVAS und Nessus könnte ausführlicher sein, OpenVAS verzichtet auf die Nennung von gefundenen Schwachstellen der Kategorie ``Informational'', bei Nessus sind die Beschreibungen der Schwachstellen und Lösungsvorschläge nur online abzurufen, das bedeutet zusätzlichen Aufwand, wenn der Auftraggeber einen gedruckten Bericht wünscht.
  Überdurchschnittlich schneiden Zed Attack Proxy, Acunetix und Netsparker ab, die alle sehr umfangreiche Beschreibungen der Schwachstellen liefern, Netsparker ist hier hervorzuheben für die zusätzliche Kategorie ``Critical'' und die besondere Kennzeichnung gesicherter Ergebnisse als ``Confirmed''.
  Überragende Ergebnisse liefern Arachni und BurpSuite Pro, Arachni sticht neben der Unterscheidung zwischen gesicherten und noch zu überprüfenden Ergenissen mit seinen ausführlichen Statistikfunktionen heraus, BurpSuite Pro mit seiner ``Confidence''-Tabelle, die die Schwachstellen noch genauer zwischen gesichert, wahrscheinlich und tentativ differenziert. Zudem bieten beide äußerst detaillierte Beschreibungen der Schwachstellen und sind trotzdem sehr übersichtlich gestaltet.
  \subsection{Scan-Geschwindigkeit}


  \section{Ranking}

  \section{Open Source oder kommerzielle WVS}

  \section{Empfehlung}
  Budget
  %Zap: Bedienbarkeit ist nicht alles


\chapter{Fazit und Ausblick}
%Die Ergebnisse dieser Thesis zeigen, dass...
%In dieser Arbeit wurden WVS auf ihre Tauglichkeit und Qualität überprüft:
%Es wurde festgestellt, dass...

%Es wurde der Unterschied zwischen Open Source und kommerziellen Scannern aufgezeigt:
%Die Fragen, die mit dieser Ausarbeitung beantwortet werden sollten...
%Scanergebnis mal 5?

%Ausblick: Diese Ausarbeitung kann in verschiedenen Richtungen erweitert werden:
%  \section{OWASP Benchmark}




\backmatter


%%%%%%%%%%%%%%%%%%%
%% create listings list
%%%%%%%%%%%%%%%%%%%
%\lstlistoflistings
%\addcontentsline{toc}{chapter}{Listings}

\cleardoublepage
\phantomsection
\addcontentsline{toc}{chapter}{Quellenverzeichnis}
\printbibliography[title=Quellenverzeichnis]

\appendix
  \chapter{Anhang}

  %\addtocontents{toc}{\protect{\vspace{3ex}}}
  %\addcontentsline{toc}{chapter}{Anhang}

  \section{Verwendete Abk"urzungen}
  \textbf{BSI} - Bundesamt für Sicherheit in der Informationstechnik

  \textbf{BSoD} - ``Blue Screen of Death'' (Kritischer Systemfehler unter Microsoft-Windows)

  \textbf{CGI} - Common Gateway Interface

  \textbf{CPU} - Central Processing Unit

  \textbf{CVE} - Common Vulnerabilities and Exposures

  \textbf{DAST} - Dynamic Application Security Testing

  \textbf{DoS} - Denial of Service

  \textbf{HTML} - Hypertext Markup Language

  \textbf{HTTP} - Hypertext Transfer Protocol

  \textbf{HTTPS} - Hypertext Transfer Protocol Secure

  \textbf{JSP} - JavaServer Pages

  \textbf{LDAP} - Lightweight Directory Access Protocol

  \textbf{NTLM} - New Technology Lan Manager

  \textbf{OS} - Operating System

  \textbf{OWASP} - The Open Web Application Security Project

  \textbf{RAM} - Random-Access Memory

  \textbf{REST} - Representational State Transfer

  \textbf{SAST} - Static Application Security Testing

  \textbf{SCA} - Static Code Analysis

  \textbf{SPA} - Single Page Application

  \textbf{SSL} - Secure Sockets Layer

  \textbf{SQL} - Structured Query Language

  \textbf{TLS} - Transport Layer Security

  \textbf{URI} - Uniform Resource Identifier

  \textbf{URL} - Uniform Resource Locator

  \textbf{WAF} - Web Application Firewall

  \textbf{WVS} - Web Application Vulnerability Scanner

  \textbf{XML} - Extensible Markup Language

  \textbf{XSS} - Cross Site Scripting
  \newpage

  \section{OWASP: The Open Web Application Security Project}
  Das ``Open Web Application Security Project'' (OWASP) ist eine unabhängige, weltweite Community mit dem Ziel, die Bedeutung der Sicherheit von Webanwendungen »sichtbar zu machen«, Fachwissen zur Entwicklung und den Betrieb sicherer Webanwendungen zu verbreiten und frei zur Verfügung zu stellen.
  OWASP ist mit keinem Technologieunternehmen verbunden, obwohl der Einsatz kommerzieller Sicherheitstechnologien unterstützt wird. Sämtliche OWASP-Instrumente, wie Dokumente, Videos, Slides, Podcasts etc. sind kollaborativ produziert worden und sind zudem kostenlos unter einer freien Lizenz verwendbar. Die OWASP Foundation ist eine Non-Profit-Organisation, die sich vier Grundwerten verschrieben hat \cite{OWASPabout}:
  \begin{itemize}
    \item Offenheit: Von den Finanzen bis zum Code ist alles radikal transparent.
    \item Innovation: OWASP fördert und unterstützt Innovationen und Experimente zur Lösung von Software-Sicherheitsherausforderungen.
    \item Globalität: Jeder auf der ganzen Welt kann sich an der OWASP-Community beteiligen.
    \item Integrität: OWASP ist eine ehrliche und aufrichtige, herstellerneutrale, globale Gemeinschaft.
  \end{itemize}
  Zudem gibt es einen Verhaltenskodex mit folgenden Prinzipien:
  \begin{itemize}
    \item Führe alle beruflichen Tätigkeiten und Pflichten in Übereinstimmung mit allen anwendbaren Gesetzen und den höchsten ethischen Grundsätzen aus.
    \item Fördere die Umsetzung von Normen, Verfahren und Kontrollen für die Anwendungssicherheit und deren Einhaltung.
    \item Bewahre eine angemessenene Vertraulichkeit gegenüber geschützter oder anderweitig sensibler Informationen, die im Rahmen einer beruflichen Tätigkeit auftreten.
    \item Übernimm die berufliche Verantwortung mit Sorgfalt und Ehrlichkeit.
    \item Kommuniziere offen und ehrlich.
    \item Vermeide Aktivitäten, die einen Interessenskonflikt hervorrufen oder anderweitig den Ruf des Arbeitgebers, den Informationssicherheitsberuf oder die Vereinigung beeinträchtigen könnten.
    \item Bewahre und verstärke unsere Objektivität und Unabhängigkeit.
    \item Weise unangemessenen Druck der von Seiten der Industrie oder anderen zurück.
    \item Verletze oder bestreite nicht absichtlich den Ruf von Kollegen, Kunden oder Arbeitgebern.
    \item Behandle jeden mit Respekt und Würde.
    \item Vermeide Beziehungen, die die Objektivität und Unabhängigkeit von OWASP  beeinträchtigen könnten.
  \end{itemize}
    \subsection{Die OWASP Top Ten}
    Eines der bekanntesten und wichtigsten Projekte von OWASP sind die OWASP Top Ten. Alle drei bis vier Jahre wird eine Liste mit den 10 häufigsten Sicherheitsrisiken für Webanwendungen veröffentlicht, um Unternehmen und Organisationen für Websicherheit zu sensibilisieren. Die Liste wird von einem weltweiten Team von IT-Sicherheitsexperten zusammengestellt, und kann durch die große Akzeptanz in der Fachwelt als Sicherheitsrichtlinie angesehen werden.

    Nachfolgend werden die aktuellen OWASP Top Ten aus dem Jahr 2017 aufgelistet \cite{OWASPtop10}:

    \subsubsection{A1: Injection}
    Injection-Schwachstellen, wie beispielsweise SQL-, OS- oder LDAP-Injection, treten auf, wenn
    nicht vertrauenswürdige Daten von einem Interpreter als Teil eines Kommandos oder einer
    Abfrage verarbeitet werden. Ein Angreifer kann Eingabedaten dann so manipulieren, dass er nicht
    vorgesehene Kommandos ausführen oder unautorisiert auf Daten zugreifen kann.

    \subsubsection{A2: Fehler in der Authentifizierung}
    Anwendungsfunktionen, die im Zusammenhang mit Authentifizierung und Sessionmanagement
    stehen, werden häufig fehlerhaft implementiert. Dies erlaubt es Angreifern, Passwörter oder
    Session-Token zu kompromittieren oder die entsprechenden Schwachstellen so auszunutzen,
    dass sie die Identität anderer Benutzer vorübergehend oder dauerhaft annehmen können.

    \subsubsection{A3: Verlust der Vertraulichkeit sensibler Daten}
    Viele Anwendungen schützen sensible Daten, wie personenbezogene Informationen und Finanzoder
    Gesundheitsdaten, nicht ausreichend. Angreifer können diese Daten auslesen oder
    modifizieren und mit ihnen weitere Straftaten begehen (Kreditkartenbetrug, Identitätsdiebstahl
    etc.). Vertrauliche Daten können kompromittiert werden, wenn sie nicht durch Maßnahmen, wie
    Verschlüsselung gespeicherter Daten und verschlüsselte Datenübertragung, zusätzlich geschützt
    werden. Besondere Vorsicht ist beim Datenaustausch mit Browsern angeraten.

    \subsubsection{A4: XML External Entities}
    Viele veraltete oder schlecht konfigurierte XML Prozessoren berücksichtigen Referenzen auf
    externe Entitäten innerhalb von XML-Dokumenten. Dadurch können solche externen Entitäten
    dazu eingesetzt werden, um mittels URI Datei-Handlern interne Dateien oder File-Shares offenzulegen
    oder interne Port-Scans, Remote-Code-Executions oder Denial-of-Service Angriffe
    auszuführen.

    \subsubsection{A5: Fehler in der Zugriffskontrolle}
    Häufig werden die Zugriffsrechte für authentifizierte Nutzer nicht korrekt um- bzw. durchgesetzt.
    Angreifer können entsprechende Schwachstellen ausnutzen, um auf Funktionen oder Daten
    zuzugreifen, für die sie keine Zugriffsberechtigung haben. Dies kann Zugriffe auf Accounts
    anderer Nutzer sowie auf vertrauliche Daten oder aber die Manipulation von Nutzerdaten,
    Zugriffsrechten etc. zur Folge haben.

    \subsubsection{A6: Sicherheitsrelevante Fehlkonfiguration}
    Fehlkonfigurationen von Sicherheitseinstellungen sind das am häufigsten auftretende Problem.
    Ursachen sind unsichere Standardkonfigurationen, unvollständige oder ad-hoc durchgeführte
    Konfigurationen, ungeschützte Cloud-Speicher, fehlkonfigurierte HTTP-Header und Fehlerausgaben,
    die vertrauliche Daten enthalten. Betriebssysteme, Frameworks, Bibliotheken und Anwendungen
    müssen sicher konfiguriert werden und zeitnah Patches und Updates erhalten.

    \subsubsection{A7: Cross-Site-Scripting (XSS)}
    XSS tritt auf, wenn Anwendungen nicht vertrauenswürdige Daten entgegennehmen und ohne
    Validierung oder Umkodierung an einen Webbrowser senden. XSS tritt auch auf, wenn eine
    Anwendung HTML- oder JavaScript-Code auf Basis von Nutzereingaben erzeugt. XSS erlaubt es
    einem Angreifer, Scriptcode im Browser eines Opfers auszuführen und so Benutzersitzungen zu
    übernehmen, Seiteninhalte verändert anzuzeigen oder den Benutzer auf bösartige Seiten
    umzuleiten.

    \subsubsection{A8: Unsichere Deserialisierung}
    Unsichere, weil unzureichend geprüfte Deserialisierungen können zu Schwachstellen in der
    Remote-Code-Execution führen. Aber auch wenn das nicht der Fall ist, können
    Deserialisierungsfehler Angriffsmuster wie Replay-Angriffe, Injections und Erschleichung
    erweiterter Zugriffsrechte ermöglichen.

    \subsubsection{A9: Nutzung von Komponenten mit bekannten Schwachstellen}
    Komponenten wie Bibliotheken, Frameworks etc. werden mit den Berechtigungen der zugehörigen
    Anwendung ausgeführt. Wird eine verwundbare Komponente ausgenutzt, kann ein solcher
    Angriff von Datenverlusten bis hin zu einer Übernahme des Systems führen. Applikationen und
    APIs, die Komponenten mit bekannten Schwachstellen einsetzen, können Schutzmaßnahmen
    unterlaufen und so Angriffe mit schwerwiegenden Auswirkungen verursachen.

    \subsubsection{A10: Unzureichendes Logging und Monitoring}
    Unzureichendes Logging und Monitoring führt zusammen mit fehlender oder uneffektiver Reaktion auf Vorfälle zu andauernden oder wiederholten Angriffen. Auch können Angreifer dadurch in Netzwerken weiter vordringen und Daten entwenden, verändern oder zerstören. Viele Studien zeigen, dass die Zeit bis zur Aufdeckung eines Angriffs bei ca. 200 Tagen liegt sowie typischerweise durch Dritte entdeckt wird und nicht durch interne Überwachungs- und Kontrollmaßnahmen.


%%%%%%%%%%%%%%%%%%%
%% declaration on oath
%%%%%%%%%%%%%%%%%%%

\addchap{Eidesstattliche Erklärung}

Hiermit versichere ich, dass ich die vorgelegte Bachelorarbeit selbstständig verfasst und noch nicht anderweitig zu Prüfungszwecken vorgelegt habe. Alle benutzten Quellen und Hilfsmittel sind angegeben, wörtliche und sinngemäße Zitate wurden als solche gekennzeichnet.

\vspace{20pt}
\begin{flushright}
$\overline{~~~~~~~~~~~~~~~~~\mbox{\BaAuthor, am \today}~~~~~~~~~~~~~~~~~}$
\end{flushright}




\end{document}
